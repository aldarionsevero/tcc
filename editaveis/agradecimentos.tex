Eu, Paulo Tada, agradeço a todos que me acompanharam durante essa jornada,
que tornaram-na uma ótima experiência com momentos bons e ruins.

À minha família, meus pais e minhas irmãs, sem eles eu não teria chegado tão
longe, são parte importante da minha vida.

Aos professores da faculdade, por todo o conhecimento e ajuda durante o curso. Especialmente
aos professores Paulo Meirelles e Hilmer Neri, pelas oportunidade de trabalhos com os
projetos que foram o diferencial para a minha formação e ao nosso orientador Renato Coral,
pela atenção e paciência que teve neste trabalho.

Aos amigos, pelos melhores momentos e experiências na faculdade. Especialmente
ao Tomáz Martins, pelas boas companhia e conversas, à Thaiane Braga, pela boa energia
e animação, ao Rodrigo Siqueira, por ser um exemplo de dedicação, à
Maria Luciene, pela companhia e ajuda nas matérias, à Garota Vanessa, pelos bons filmes,
músicas e ótimas distrações.

À minha primeira equipe de trabalho e amigos de MDS e GPP, João, Beatriz, Larissa, Euler,
Levi, Átila, e Maxwell, por me fazer perceber que era isso que eu queria fazer na vida.

Aos amigos do laboratório LAPPIS, por todo o conhecimento e diversão no trabalho.
Nenhuma equipe vai ser tão boa como vocês.

Eu, Lucas Severo agradeço à todos que me apoiaram, incentivaram e ajudaram
durante essa graduação, que começou em eletrônica e passou para Engenharia de 
Software.

À minha mãe, que sempre me deu suporte, conselhos e apoio. A minha tia Rita, que
sempre esteve do meu lado em discussões com minha mãe. Aos meus padrinhos, que
sempre me incentivaram e me acolhem. Ao meu tio Padre Antônio, que desde o início
me ajudou em dificuldades e me ofereceu palavras de sabedoria para continuar.
Enfim, a toda minha família, que em sua união, sempre esteve ao meu lado.

Agradeço também aos meus amigos, em especial Pablo Alejandro, Matheus Pimenta, 
Daniel Monteiro eRodrigo Siqueira que sempre foram exemplo de dedicação, estiveram comigo em
momento de dificuldade e de alegria, e sempre me apoiaram nessa jornada e mudança
de curso.

À minha namorada, que me deu forças pra continuar, sempre foi exemplo de esforço
e dedicação, e minha parceira em decisões importantes.

Aos companheiros que surgiram na minha vida após mudar para Engenharia de Software,
membros do laboratório LAPPIS e dos diversos grupos formados em disciplinas. Diversas
experiências e conhecimentos que tenho devo à essas pessoas.

Aos Professores, que foram fonte de inspiração e de conhecimento. Ao professor
Edson, que foi o primeiro professor que me apresentou a um projeto de iniciação
cientifica, me inspirou a ser um profissional melhor, e me orientou em diversos
aspectos. Ao professor Marcelino, que despertou em mim a vontade de empreender
e mudar um pouco do cenário empreendedor brasileiro. Ao professor Paulo, que
também me orientou em diversos aspectos, me apresentou a diversas oportunidades,
e possibilitou que tivesse contato com diversas pessoas que me fizeram um
profissional melhor. E ao professor Renato, por sua prestatividade, dedicação
e atenção ao nos orientar.
