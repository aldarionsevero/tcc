\section{Funcionalidade de Ajuda}
\label{sec:help}

A funcionalidade de ajuda trata basicamente do comando que imprime na tela um resumo
dos outros comandos e suas opções. Já que o Cupper segue os padrões de ferramentas
CLI (\textit{Command Line Interface}), ter um comando de ajuda se faz necessário
e útil.

Para montar a estrutura de comandos foi utilizado o Thor, uma gem para criação
de CLIs. Ele facilita o parseamento de entradas e a documentação de cada comando criado.
O próprio thor facilita a criação dos comandos de ajuda pra cada um dos comandos
criados. Basta documentar cada método que será um comando com templates de cada
comando e suas descrições como na primeira linha do exemplo do Código~\ref{code:thor}

\noindent\begin{minipage}{\textwidth}
  \lstinputlisting[language=Ruby, frame=single, label=code:thor, caption="Exemplo de descrição para documentação e ajuda"]{conteudo/code/thor.rb}
\end{minipage}\hfill

\subsection{Ajuda geral}
Utilizando das funcionalidades do Thor, a ajuda geral imprime a lista de todos
os comandos base, como utilizálos e suas descrições. O que o Thor faz é basicamente
imprimir todos os \textit{desc}`s presentes nós métodos de comando.

Um exemplo de saída pode ser visto no Código~\ref{code:thorsaida}.

\noindent\begin{minipage}{\textwidth}
  \lstinputlisting[language=Bash, frame=single, label=code:thorsaida, caption="Exemplo de saída de ajuda geral do Thor"]{conteudo/code/thorsaida.sh}
\end{minipage}\hfill


\subsection{Ajuda específica}
Semelhante a ajuda geral, mas para um comando específico. O Thor somente imprime
na tela o \textit{desc} de um comando específico como pode ser visto no Código~\ref{code:thorsaida2}.

\noindent\begin{minipage}{\textwidth}
  \lstinputlisting[language=Bash, frame=single, label=code:thorsaida2, caption="Exemplo de saída de ajuda específica do Thor"]{conteudo/code/thorsaida2.sh}
\end{minipage}\hfill
