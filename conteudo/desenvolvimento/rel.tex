\section{Funcionalidade gerar relatório intermediário}
\label{sec:rel}


Essa funcionalidade trata da geração de um relatório intermediário antes da 
receita em si, para diagnóstico do sistema sendo extraído e para debug do 
Cupper. Ela prevê extrair informações adicionais de hardware, de sistema, e de
diversos aspectos como descrito na seção~\ref{sec:cam-amb}. Infelizmente essa
funcionalidade não foi implementada.

\subsection{\textit{plugins} de Extração}
Para a extração dos atributos de cada camada descrita na seção~\ref{sec:cam-amb},
diversos \textit{plugins} devem ser implementados, e vários \textit{plugins} já prontos do Ohai
devem ser utilizados.

A necessidade ou não da implementação, e quais comandos acessam a informação 
que se deseja podem ser vistos nas Tabelas~\ref{tab:atrhard},~\ref{tab:hdmount},~\ref{tab:so},~\ref{tab:app},~\ref{tab:config} e~\ref{tab:service}.

\subsection{Geração do Relatório}
A saída do relatório será em um formato JSON (\textit{JavaScript Object Notation}),
assim como a saída oficial da gem do Ohai quando se executa em um ambiente. Será,
muito semelhante a execução da gem do Ohai, mas além de utilizar os \textit{plugins} nativos
do Ohai, também utilizará todos os \textit{plugins} custom do cupper e do Ohai.

\subsection{Mudanças e Justificativas}
Essa funcionalidade não foi implementada devido a priorização realizada e andamento
do projeto. Funcionalidades que traziam mais valor para o usuário, como 
gerar a receita com os recursos Chef em si tiveram mais prioridade e foram implementadas
primeiro.
