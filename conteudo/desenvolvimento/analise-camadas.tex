\section{Camadas de Ambiente}
\label{sec:cam-amb}

As camadas apresentam um conjunto de aspectos do sistema e contém as
informações necessárias que definem o comportamento específico para o
estado desejado daquele ambiente. Os atributos de cada camada são variantes
que são consideradas para o \textit{deployment} de uma aplicação. Sendo assim, as
camadas são dependentes para o nível de compatibilidade da configuração.

\subsection{Levantamento de Camadas de Ambiente}
As seguintes camadas foram definidas para representar o estado de configuração do
sistema:
\begin{itemize}
  \item \textit{\textbf{Hardware}}: definições físicas onde o sistema foi implantado.
    (Arquitetura, memória, espaço em disco, etc);
  \item \textit{\textbf{Operation System}}: definições do sistema operacional
    implantado. Distribuição, arquitetura, versão, etc;
  \item \textit{\textbf{Application}}: definições das aplicações instaladas.
    (Aplicações instaladas, dependências, etc);
  \item \textit{\textbf{Configuration}}: definições das configurações das
    aplicações. (Especificações de implantação de aplicação, arquivos
    de configuração);
  \item \textit{\textbf{Service}}: definições dos serviços daemon que estão em
    funcionamento no sistema.
  \item \textit{\textbf{Custom}}: definições criadas especificamente para o
    sistema sem uma forma padrão conhecida.
\end{itemize}

\subsubsection{\textit{Hardware}}
\label{sec:cam-hard}

A camada de \textit{Hardware} contém as definições físicas do sistema.
As informações são referentes as configurações físicas da máquina, como
CPU, arquitetura, memória, espaço em disco, particionamento, etc. O informe desses atributos
são utilizados para a definição da base do ambiente, ou seja, o sistema
operacional e as aplicações podem ter diferentes desempenhos a partir das
configurações de hardware e/ou apresentar comportamentos inesperados no sistema.
Além das aplicações terem seus requisitos mínimos, algumas são desenhadas
para um tipo específico de arquitetura.

A tabela~\ref{tab:atrhard} mostra diversas informações de hardware possíveis de serem extraídas
do sistema a partir de comandos do sistema operacional e também da ferramenta Ohai, 
dependência prevista para a implementação do Cupper (como descrito na 
seção~\ref{sec:tec}). Nem todas essas informações serão úteis para a 
implementação, mas nessa seção fazemos um levantamento geral antes de fazer uma
seleção.

\begin{table}[H]
\centering
\caption{Levantamento dos atributos de Hardware}
\label{tab:atrhard}
\begin{tabular}{|l|l|l|l|l|}
\hline
Atributo & Origem & Relevancia & Dificuldade de integrar & Será utilizado? \\ \hline
         &        &            &                         &                 \\ \hline
         &        &            &                         &                 \\ \hline
         &        &            &                         &                 \\ \hline
         &        &            &                         &                 \\ \hline
         &        &            &                         &                 \\ \hline
\end{tabular}
\end{table}

A camada de hardware será uma das camadas a serem analisadas, e a seleção de 
seus atributos irão seguir os critérios descritos na seção~\ref{sec:defcritcamada}.

{\color{red} colocar uma tabela com saida dos comandos lscpu, lshw, hwinfo, lspci
lsscsi, lsusb, lsblk, df, mount, free, e ohai}

{\color{red} explicar de onde vem cada informação, e cada comando}

{\color{red} tirar o que for redundante}

%TODO: adicionar as informações que serão coletadas.

\subsubsection{\textit{Operation System}}
\label{sec:cam-os}

A segunda camada de extração de dados é a do Sistema Operacional. 
Cada sistema tem a sua particularidade com relação aos gerenciamento dos 
recursos de hardware e administração de processos. Portanto é necessário a 
construção de uma arquitetura flexível, que contenha os aspectos em comum aos
sistemas operacionais, e modular para que seja possível escalar a utilização 
da ferramenta para mais Sistemas Operacionais.

Inicialmente a ferramenta Cupper irá abordar dois sistemas operacionais, 
ambos baseados em GNU/Linux: Archlinux e Debian.

A tabela~\ref{} mostra informações a respeito do Kernel e da distribuição
analisada extraídas a partir de comandos do próprio sistema ou do Ohai.


{\color{red} tabela com infos de distro, versões, pkg manager tool, e kernel do ohai}

{\color{red} detectar qual init system é usado verificando existência de diretórios
e outros métodos}

{\color{red} extrair qual sistema de segurança é usado verificando existência
do apparmour, selinux ou grsecurity}

{\color{red} explicar de onde veio cada dado}

%TODO: a seleção de camadas e profundidade de análise serão postas dentro das
%   especificações de cada camada.

