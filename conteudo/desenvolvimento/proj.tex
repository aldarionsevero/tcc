\section{Funcionalidade gerar diretório de projeto}
\label{sec:proj}

Essa funcionalidade cria a estrutura de diretório para que o Cupper gere as 
receitas Chef e leia o seu arquivo de configuração, o Cupperfile.

Como exibido na saída~\ref{code:thorsaida2}, o básico dessa funcionalidade é, bem
simples. Um diretório com o nome escolhido vai ser criado, dentro dele um diretório
para as cookbooks, um arquivo oculto para listar arquivos sensíveis, e um 
Cupperfile para as definições que o usuário define para a extração.

\subsection{Definição de ambiente}
Existem algumas classes importantes no código do Cupper que são relacionadas a
definição do ambiente de execução. Umas deas é a classe do \textit{Cupperfile} e outra
é o \textit{Environment}.

A classe de \textit{Environment} é responsável por checar se o ambiente é um
ambiente válido com a presença de um Cupperfile, definir o diretório do ambiente
(\textit{root\underline{ }path}) e instanciar um \textit{Cupperfile}. A classe \textit{Cupperfile} carrega as configurações definidas no arquivo Cupperfile.

\subsection{Problemas encontrados}
Devido a priorização realizada, e com o andamento do projeto, o carregamento de
configurações adicionais vindas do Cupperfile tinha sido tirado do backlog do
trabalho.

Essa decisão seguiu por algum tempo até que algumas falhas fizeram o Cupperfile
ser indispensável para deixar o Cupper funcionando para mais casos.

\subsubsection{Pacotes com repositório fora do \textit{source list} para o 
gerenciador de pacotes}
Esse problema surgia quando um pacote que foi instalado manualmente, ou que
teve seu reposotório removido do \textit{source list} era extraído usando o Cupper,
mas ao executar a receita em um novo ambiente, a instalação falhava, já que 
o gerenciador de pacotes não encontrava a fonte do pacote.



