\section{Funcionalidade gerar diretório de projeto}
\label{sec:proj}

Essa funcionalidade cria a estrutura de diretório para que o Cupper gere as 
receitas Chef e leia o seu arquivo de configuração, o Cupperfile.

Como exibido na saída~\ref{code:thorsaida2}, o básico dessa funcionalidade é, bem
simples. Um diretório com o nome escolhido vai ser criado, dentro dele um diretório
para as cookbooks, um arquivo oculto para listar arquivos sensíveis, e um 
Cupperfile para as definições que o usuário define para a extração.

\subsection{Definição de ambiente}
Existem algumas classes importantes no código do Cupper que são relacionadas a
definição do ambiente de execução. Umas deas é a classe do \textit{Cupperfile} e outra
é o \textit{Environment}.

A classe de \textit{Environment} é responsável por checar se o ambiente é um
ambiente válido com a presença de um Cupperfile, definir o diretório do ambiente
(\textit{root\underline{ }path}) e instanciar um \textit{Cupperfile}. A classe \textit{Cupperfile} carrega as configurações definidas no arquivo Cupperfile.

\subsection{Problemas encontrados}
Devido a priorização realizada, e com o andamento do projeto, o carregamento de
configurações adicionais vindas do Cupperfile tinha sido tirado do backlog do
trabalho.

Essa decisão seguiu por algum tempo até que algumas falhas fizeram o Cupperfile
ser indispensável para deixar o Cupper funcionando para mais casos.

\subsubsection{Pacotes com repositório fora do \textit{source list} para o 
gerenciador de pacotes}
Esse problema surgia quando um pacote que foi instalado manualmente, ou que
teve seu reposotório removido do \textit{source list} era extraído usando o Cupper,
mas ao executar a receita em um novo ambiente, a instalação falhava, já que 
o gerenciador de pacotes não encontrava a fonte do pacote.

Já que isso não é o comportamento esperado, mas pode ser uma escolha do usuário
ter pacotes de outras fontes, foi criado um atributo para que o usuário defina no
Cupperfile e decida se quer que se extraia pacotes que tem fonte em repositórios
fora do \textit{source list}. O exemplo de Cupperfile que é criado pelo comando
\textit{cupper create project} tem exemplos desses atributos e explicação, como
no código~\ref{code:sourcelesspackages.rb}.

\noindent\begin{minipage}{0.7\textwidth}
  \lstinputlisting[language=Ruby, label=code:sourcelesspackages.rb, caption="Definição do atributo de \textit{sourceless packages}"]{conteudo/code/sourcelesspackages.rb}
\end{minipage}\hfill

\subsubsection{Pacotes já instalados no ambiete novo com versões menores}
Os pacotes já instlados no ambiente novo em que a receita vai rodar podem ter versões
maiores do que a dos pacotes extraídos do ambiente original. Se a receita for
montada sem nenhum tipo de tratamento, a sua execução vai quebrar no momento
em que ela tentar fazer esse \textit{downgrade}.

Foi criado então um atributo para o Cupperfile que deixa o usuário definir
se o Cupper vai forçar os \textit{downgrades} ou se vai ignorar os recursos
Cheff de pacotes que caírem nessa situação, passando \textit{action :nothing}
para esses. O exemplo vindo do Cupperfile com sua explicação pode ser visto no
código~\ref{code:allowdowngrade}

\noindent\begin{minipage}{0.7\textwidth}
  \lstinputlisting[language=Ruby, label=code:allowdowngrade, caption="Definição do atributo de \textit{allow downgrade}"]{conteudo/code/allowdowngrade.rb}
\end{minipage}\hfill

\subsubsection{Arquivos de configurações sensíveis com potencial de qubrar o sistema}
Todos os arquivos de configuração do \textit{/etc} estavam sendo copiados para
o novo sistema via receita. Isso incluia arquivo de \textit{sudoers}, \textit{/etc/group}
entre outros que tem potencial de quebrar o acesso ao sistema caso ocorra algum erro.

Para evitar esse problema, mas deixar também como opção para o usuário, mais um
atributo para o Cupperfile foi criado, definindo se arquivos sensíveis vão ser
copiados ou não. O exemplo vindo do Cupperfile com sua explicação pode ser visto no
código~\ref{code:sensiblefiles}

\noindent\begin{minipage}{0.7\textwidth}
  \lstinputlisting[language=Ruby, label=code:sensiblefiles, caption="Definição do atributo de \textit{sensible files}"]{conteudo/code/sensiblefiles.rb}
\end{minipage}\hfill

 
\subsection{Mudanças e Justificativas}
O escopo tinha sido alterado, e o carregamento de configurações do Cupperfile
tinha sido removido do backlog. Mas esse carregamento se mostrou necessário
e acabou voltando para o desenvolvimento.

Algumas dessas definições do Cupperfile foram reativas, e implementadas de acordo
com o surgimento da necessidade de implementar. Outras tinham sido previstas, mas
não prioritárias, como a de \textit{allow downgrade}.

O que fez essa abordagem voltar a ser indispensável foi o fato de que o Cupper
não pode gerar uma receita que quebra o acesso do usuário a um sistema, ou pior,
que inutilize o sistema.
