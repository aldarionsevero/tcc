\section{Funcionalidade de extrair grupos}
\label{sec:groups}

\subsection{Atributos}

Para a ter a funcionalidade mínima esperada desse recurso, definimos que iríamos precisar
extrair os seguintes atributos:

\begin{itemize}
    {\itshape\item gid}: Identificador do grupo;
    {\itshape\item members}: Identificar membros do grupo;
\end{itemize}


\begin{itemize}
    {\itshape\item append}: Definir que os membros podem ser adicionados e removidos;
    {\itshape\item action}: Definir ação de criar o grupo;
\end{itemize}

\subsection{\textit{Plugin} de Extração}
Não foi necessário criar um \textit{plugin} para o Ohai que extraísse esses atributos
referentes à grupos já que o Ohai já possui um. O nome \textit{plugin} usado é 
\textit{passwd} e ele provê informações de grupos, usuários, permissões entre outras.

\subsection{Geração de bloco de receita}

Para a criação da receita que cria os grupos em um novo ambiente, foi definido
como seria o template dos blocos de recurso de grupo na sintaxe das receitas Chef.
Um exemplo pode ser visto no código~\ref{code:groupresource}.

\noindent\begin{minipage}{\textwidth}
  \lstset{style=shell}
  \lstinputlisting[language=Ruby, frame=single, label=code:groupresource, caption="Bloco de recurso de grupo Chef"]{conteudo/code/groupresource.rb}
\end{minipage}\hfill

Se uma receita com esse recurso for executada em um ambiente, o grupo \textit{www-data}
irá ser modificado com um novo membro \textit{maintenance}.

Para gerar então diversos desses blocos na receita final o template do código~\ref{code:grouptemplate} foi definido.

\noindent\begin{minipage}{\textwidth}
  \lstset{style=shell}
  \lstinputlisting[language=Ruby, frame=single, label=code:grouptemplate, caption="Template para os blocos de recursos de grupo Chef"]{conteudo/code/grouptemplate.rb}
\end{minipage}\hfill

Dessa forma, como visto acima, para criar um bloco de receita de recursos Chef é
necessário saber qual vai ser o template da escrita desse bloco e qual \textit{plugin}
o coletor vai utilizar para a extração.

\subsection{Problemas encontrados}

Essa funcionalidade de extração de grupos não tinha sido prevista inicialmente.
Como citado na Seção~\ref{sec:users}, ela nasceu de um problema com a criação de usuários
de outra funcionalidade. Sem a organização desses usuários em grupos, o Chef não conseguia
executar a receita.

% mapear quais problemas bota aqui e quais bota em users mesmo
