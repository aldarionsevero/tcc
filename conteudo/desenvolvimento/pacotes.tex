\section{Funcionalidade de extrair pacotes}
\label{sec:pacotes}

\subsection{Atributos}
Para ter a funcionalidade mínima esperada desse recurso, definimos que iríamos
precisar extrair os seguintes atributos:

\begin{itemize}
  \item \textit{version}: especifica a versão do pacote;
  \item \textit{action}: define a ação que deve ser tomada
    (instalar, remover, reconfigurar, etc);
\end{itemize}

\subsection{\textit{Plugin} de Extração}

O Ohai provê um plugin nativo que coleta todos os pacotes instalados. O suporte
das plataformas são: Debian, Redhat, Fedora e OpenSUSE\@. Os dados coletados pelo
plugin de cada pacote são: nome do pacote e versão.

Para a plataforma Debian, o plugin utiliza a ferramenta \texttt{dpkg-query} disponível
por padrão nas distro Debian \textit{based}. O \texttt{dpkg-query} retorna uma lista de todos os pacotes
instalados no ambiente que é interpretada pelo plugin e transformada em um JSON
construído com a estrutura \texttt{``package\_name'' => \{ ``version'' => ``version\_number'' \}}
como mostrada no Código~\ref{code:json_pkg}.

\noindent\begin{minipage}{\textwidth}
  \lstset{style=shell}
  \lstinputlisting[frame=single,
    label=code:json_pkg,
    caption="Saída JSON do plugin Ohai \textit{packages}"]{conteudo/code/json_pkg.json}
\end{minipage}\hfill

Para as plataformas Redhat, Fedora e OpenSUSE, o plugin utiliza a ferramenta \texttt{rpm}
disponível por padrão nas distros Redhat \textit{based}. O Cupper ainda não oferece suporte
para distros Redhat \textit{based}, portanto não foram feitos testes para esse tipo de ambiente.

Não há suporte para a plataforma Arch Linux, portanto um plugin foi desenvolvido
para extrair as mesmas informações. A extração utiliza a ferramenta \texttt{pacman} disponível
nas distros Arch Linux. O \texttt{pacman} retorna uma lista de pacotes instalados no ambiente
que é interpretada pelo plugin e transformada em um JSON construído com a estrutura
\texttt{"package\_name" => \{ "version" => "version\_number" \}} como mostrada no
Código~\ref{code:json_pkg_pacman}.

\noindent\begin{minipage}{\textwidth}
  \lstset{style=shell}
  \lstinputlisting[frame=single,
    label=code:json_pkg_pacman,
    caption="Saída JSON do plugin Ohai \textit(pacman)"]{conteudo/code/json_pkg_pacman.json}
\end{minipage}\hfill

\subsection{Bloco de Receita Gerado}

O JSON gerado pela extração é utilizado para construir a receita responsável pela instalação dos
pacotes. É utilizado o recurso Chef \textit{package} para instalação de cada pacote com os dois atributos
\textit{version} e \textit{action} (Código~\ref{code:pkg_recipe}). \textit{Action} é colocado \textit{\textit{:install}}
como padrão para todos os pacotes.

\noindent\begin{minipage}{\textwidth}
  \lstset{style=shell}
  \lstinputlisting[
    language=Ruby,
    frame=single,
    label=code:pkg_recipe,
    caption="Exemplo de receita gerada pela extração de pacotes."]{conteudo/code/pkg_recipe.rb}
\end{minipage}\hfill

Considera-se que o pacote irá cuidar da suas próprias dependências quanto a instalação.
Sendo assim, os pacotes que são dependências de outros não são inclusos na receita. Exemplo, se o pacote
\textbf{A} contém as dependências \textbf{B} e \textbf{C} elas não serão inclusas na receita, pois o pacote \textbf{A} irá realizar a instalação.

\subsection{Problemas Encontrados}

O método de extração dos pacotes instalados não realiza nenhuma checagem para
saber se o pacote continua disponível no repositório de pacotes remotos ou
se ao menos tem um repositório remoto associado. Isso causou falha na realização
dos testes. Tanto os pacotes instalados manualmente, por meio de download do
\textit{source}, quanto os pacotes que não estão mais disponíveis no repositório remoto
não eram identificado durante a configuração do novo ambiente.

Já que isso não é o comportamento esperado, mas pode ser uma escolha do usuário
ter pacotes de outras fontes, foi criado um atributo para que o usuário defina no
Cupperfile e decida se quer extrair pacotes que tem fonte em repositórios
fora do \textit{source list}. O exemplo de Cupperfile que é criado pelo comando
\textit{cupper create project} tem exemplos desses atributos e explicação, como
no Código~\ref{code:sourcelesspackages.rb}.

\noindent\begin{minipage}{\textwidth}
  \lstset{style=shell}
  \lstinputlisting[
    language=Ruby,
    label=code:sourcelesspackages.rb,
  caption="Definição do atributo de \textit{sourceless packages}"]{conteudo/code/sourcelesspackages.rb}
\end{minipage}\hfill

A receita de pacote especifica para todos os pacotes a versão que foi extraída do ambiente.
Ao realizar a configuração do novo ambiente, é possível que o pacote já esteja instalado,
mas a sua versão esta à frente da definida na receita. Quando isso ocorre, dois comportamentos
podem aparecer: é feito o \textit{downgrade} do pacote ou ocorre quebra no \textit{downgrade} das
dependências do pacote.

Foi criado o atributo \texttt{allow\_downgrade} para o Cupperfile que possibilita
o usuário definir se a receita gerada irá conter atributos que force instalar
a versão antiga (Código~\ref{code:allowdowngrade}). O método de realizar isso
é por meio da desinstalação completa do pacote, que inclui as dependências,
e a instalação do pacote antigo.

\noindent\begin{minipage}{\textwidth}
  \lstset{style=shell}
  \lstinputlisting[
    language=Ruby,
    label=code:allowdowngrade,
    caption="Definição do atributo de \textit{allow downgrade}"]{conteudo/code/allowdowngrade.rb}
\end{minipage}\hfill
