\section{Funcionalidade de extrair serviços}
\label{sec:services}

\subsection{Atributos}

O recurso de serviços do Chef prevê os seguintes atributos, alguns obrigatórios, outros não:

\begin{itemize}
{\itshape\item init\_command}
{\itshape\item pattern}
{\itshape\item priority}
{\itshape\item provider}
{\itshape\item reload\_command}
{\itshape\item restart\_command}
{\itshape\item service\_name}
{\itshape\item start\_command}
{\itshape\item stop\_command}
{\itshape\item subscribes}
{\itshape\item supports}
{\itshape\item timeout}
{\itshape\item action}
\end{itemize}

Para a ter a funcionalidade mínima esperada desse recurso, definimos que iríamos precisar
extrair os seguintes atributos:


\begin{itemize}
    {\itshape\item provider}: Identificar qual \textit{Init System} está gerenciando
esse serviço;
    {\itshape\item service\_name}: Identificar o nome do serviço a
ser levantado;
    {\itshape\item action}: Identificar ação, padrão é restart;
\end{itemize}

\subsection{\textit{Plugin} de Extração}
Para a extração desses atributos referentes aos serviços, foi criado um \textit{plugin} do
Ohai que, a partir do comando \textit{systemctl} lista todos os serviços do ambiente
e extrai os atributos de \textit{service\_name}, \textit{provider}
e \textit{action}.

Esses atributos são coletados e para cada serviços extraído um objeto de serviço
é criado para ser iterado no template posteriormente.

\subsection{Geração de bloco de receita}
Para a geração da receita que levanta esses serviços em outro ambiente, foi 
definido como seria o template dos blocos de recurso de serviço na sintaxe 
\textit{Ruby like} própria das receitas Chef. Um exemplo pode ser visto no Código~\ref{code:serviceresource}.
% TODO muito repetitivo... dar uma limpada

\noindent\begin{minipage}{\textwidth}
  \lstset{style=shell}
  \lstinputlisting[language=Ruby, frame=single, label=code:serviceresource, caption="Bloco de recurso de serviço Chef"]{conteudo/code/serviceresource.rb}
\end{minipage}\hfill

Esse exemplo, se executado em um ambiente, diz para o
gerenciador de serviços (\textit{Init System}) dar um start no serviço do tomcat.

Dessa forma, para gerar diversos desses blocos na receita final, temos o template
do Código~\ref{code:servicetemplate}.

\noindent\begin{minipage}{\textwidth}
  \lstset{style=shell}
  \lstinputlisting[language=Ruby, frame=single, label=code:servicetemplate, caption="Template para os blocos de recursos de serviço Chef"]{conteudo/code/servicetemplate.rb}
\end{minipage}\hfill

Então para gerar um bloco de receita de recursos de serviço Chef é necessário
saber como vai ser o template da escrita desse bloco e qual \textit{plugin} o
coletor vai utilizar para a extração.

No Código~\ref{code:servicerecipe} é possível ver a chamada para a inicialização e escrita de um
bloco de receita de serviços. Todos os parâmetros necessários são passados para
o construtor da classe e então o método create que escreve a receita é chamado.
O parâmetro \textit{collector} passa um coletor genérico que carregará todos os
atributos desse e de outros recursos, o parâmetro \textit{`\_services'} é uma
string com o nome do template a ser usado para escrever a receita, e o ultimo
parâmetro é uma string que indica qual \textit{plugin} o coletor deverá utilizar.

\noindent\begin{minipage}{\textwidth}
  \lstset{style=shell}
  \lstinputlisting[language=Ruby, frame=single, label=code:servicerecipe, caption="Chamada para inicialização e escrita de um bloco de receita"]{conteudo/code/servicerecipe.rb}
\end{minipage}\hfill



\subsection{Problemas encontrados}

\subsubsection{Distinção entre serviços instalados e serviços rodando}
A primeira abordagem utilizada para a extração dos atributos referentes a esse
recurso não previa a distinção entre serviços instalados e serviços que estão
ativos ou estão rodando. A extração ocorria sem nenhum problema, e todos os serviços
instalados eram extraídos. Mas ao executar a receita em um novo ambiente, diversos
serviços falhavam ao tentar se inicializar.

Serviços que já estavam falhando ou já estavam inativos no ambiente sendo extraído
também iriam falhar no ambiente novo a ser gerado, e isso teve que ser tratado.


Para solucionar esse problema só serviços que tem status \textit{loaded, active e running}
passaram a ser extraídos.\

\subsubsection{Distinção entre diferentes \textit{Init Systems}}
Outro problema similar foi o de não verificar qual Init System gerenciava cada
serviço. Já que o recurso de serviço do Chef considera por padrão que os serviços
são \textit{systemd}, quando algum serviço não era, falhas aconteciam.

A extração passou então a extrair o atributo \textit{provider} que ao ser passado
para o recurso de serviço do Chef, solucionava esse problema.

\subsubsection{Distinção entre \textit{systemd units}}
O comando utilizado para que o \textit{plugin} do Ohai extraia os serviços não
lista somente serviços em si, já que ele lista todo tipo de \textit{systemd units}
como \textit{mounts, devices, targets, paths}. Esse comportamento não é ideal
tanto por gerar falhas quanto por não fazer parte do escopo interagir com esses
outros tipos de \textit{systemd units}.

O \textit{plugin} passou então a parsear as unidades e só pegar os que são serviços.

\subsection{Mudanças e Justificativas}
Nenhuma mudança com relação ao que era previsto para a funcionalidade de extrair
atributos relacionados a serviços foi feita. As mudanças que foram feitas com relação
às abordagens de desenvolvimento se devem somente ao fato de fazer a
funcionalidade ter o comportamento esperado.
