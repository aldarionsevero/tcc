\section{Funcionalidade de Extrair Links}
\label{sec:links}

\subsection{Atributos}
Para o recurso Chef de \texttt{link} foi definido os seguintes
atributos para a extração:

\begin{itemize}
  \item \textit{group}: define o grupo que pertence o link;
  \item \textit{owner}: dono do link, referente ao usuário no ambiente;
  \item \textit{mode}: permissões do link;
  \item \textit{to}: caminho para o arquivo real;
\end{itemize}

\subsection{\textit{Plugin} de Extração}

O mesmo plugin utilizado para recuperar arquivos de configuração descritos
na Seção~\ref{sec:files} também é usado para extrair \textit{links}. Isso é possivel
pelo atributo \textit{type} coletado pelo plugin. Nele contem, além do tipo do arquivo,
o caminho para o arquivo real. Os outros atributos são comuns entre arquivos e
\textit{links}, que são coletados e tratados da mesma forma. O JSON gerado segue a
mesma estrutura dos arquivos como mostra o Código~\ref{code:json_links}.

\noindent\begin{minipage}{\textwidth}
  \lstset{style=shell}
  \lstinputlisting[
    label=code:json_links,
    caption="Saída JSON do plugin Ohai \textit{files} com exemplo de link."]{conteudo/code/json_links.json}
\end{minipage}\hfill

\subsection{Bloco de Receita Gerado}

A receita gerada usa o recurso \texttt{links} (Código~\ref{code:links_recipe}) com os
atributos para cada \textit{link} gerado.

\noindent\begin{minipage}{\textwidth}
  \lstset{style=shell}
  \lstinputlisting[
    label=code:links_recipe,
    caption="Exemplo de receita gerada pela extração de links."]{conteudo/code/links_recipe.rb}
\end{minipage}\hfill

O atributo \textit{mode} é irrelevante neste caso. Todos os \textit{links} gerados tem a mesma permissão
777, pois o que determina a leitura, escrita ou execução do \textit{link} é o arquivo para o qual ele
direciona. Sendo assim, esse atributo é irrelevante e será removido em atualizações futuras.

\subsection{Problemas Encontrados}

Durante os testes de configuração dos novos ambientes, os \textit{links} continham caracteres
especias que não eram reconhecidos pelo \textit{encode} padrão do Ruby causando erro.
Esse tipo de texto requer um tratamento especial dentro do código, mas as tentativas
de correção desse problema não tiveram resultados positivos. A alternativa que encontramos
para contornar temporariamente esse problema é a remoção de qualquer link ou arquivo que
não estão nos formatos de \textit{encode} padrão suportados pela ferramenta.

