\section{Escopo de Implementação}
\label{sec:escopo}
Nessa seção serão descritas as funcionalidades e a estrutura inicial da gem.

\subsection{Funcionalidades}

Já que o \textit{Cupper} é uma ferramenta de linha de comando, suas funcionalidades
estão diretamente relacionadas aos subcomandos e suas opções. Dessa forma os
comandos chave da aplicação irão ser listados nessa seção.

\subsubsection{Mostrar Ajuda}

Como diversos outras ferramentas de linha de comando, o \textit{Cupper} deve possuir
um subcomando que imprime no terminal os principais subcomandos e algumas de 
suas opções.

\subsubsection{Gerar Cookbook}

Essa é a funcionalidade mais importante do \textit{Cupper}, e é a motivação
principal para implementá-lo e utilizá-lo. A figura~\ref{fig:cupper_geral} mostra
o fluxo básico do funcionamento.

\subsubsection{Gerar Relatório}

É possível também utilizar o \textit{Cupper} para gerar um relatório mais
completo do que o \textit{Ohai} geraria, considerando os \textit{plugins} da
família \textit{Cupper} para o \textit{Ohai}.

Esse relatório pode ser utilizado com fins de diagnóstico, ou como uma fase 
intermediária à criação de \textit{Cookbooks}, para entender o que será finalmente 
escrito nas \textit{Cookbooks} e também por motivos de \textit{debug}.

\subsubsection{Gerar Diretório Inicial para Configuração}

Essa funcionalidade é referente ao subcomando que irá gerar a estrutura de
diretórios padrão para os arquivos de configuração do \textit{Cupper} e para
a inserção de \textit{plugins}.

\subsubsection{Listar \textit{Plugins} do \textit{Ohai}}

Funcionalidade referente ao subcomando que lista os \textit{pligins} do \textit{Ohai}
para simples conferencia.

\subsubsection{Listar \textit{Plugins Custom} do \textit{Cupper}}

Semelhante à funcionalidade anterior, mas lista \textit{Plugins Custom} do 
\textit{Cupper}.

\subsection{Estrutura e módulos}

A figura~\ref{fig:cupper-detail} mostra a Estrutura Inicial Planejada para os 
Módulos do \textit{Cupper}.

\begin{figure}[H]
  \centering
  \includegraphics[width=1.05\textwidth]{figuras/cupper_detail.eps}
  \caption{Estrutura Inicial Planejada para os Módulos do Cupper}
  \label{fig:cupper-detail}
\end{figure}

% \subsection{Criação de \textit{Cookbooks}}
