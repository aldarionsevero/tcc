\section{Recursos Chef}
\label{sec:lev-rec}

Como descrito ne seção \ref{sec:chef}, o Chef é disposto em vários
componentes. O foco principal do Cupper é a criação de \textit{cookbooks},
sendo assim os recursos levantados são referente a composição interna
dos \textit{cookbooks}.

Os \textit{cookbooks} contém os seguintes componentes: \textit{attributes, recipes, definitions,
files, libraries, custom resourses, metadata, resources e providers, templates
cookbook versions}~\cite{chefdoc:2017}.

Para o funcionamento de um \textit{cookbook}, com uma estrutura mínima, é necessário que tenha
ao menos um \textit{recipe default} definido. O exemplo \ref{code:tree} mostra essa
estrutura mínima necessária para realizar a configuração do \textit{cookbook app}.

\noindent\begin{minipage}{.9\textwidth}
  %\lstset{style=shell}
  \lstinputlisting[language=Bash, label=code:tree, caption="Estrutura mínima de um cookbook"]{conteudo/code/minimal_tree}
\end{minipage}\hfill

As subseções seguintes explicam os componentes dos \textit{cookbooks} e quais serão gerados
pelo Cupper.

\subsection{\textit{Attributes}}
\label{sec:lev-rec-att}

Os \textit{attributes} são os detalhes das especificações do \textit{node}. Definem~\cite{chefdoc:2016}:

\begin{itemize}
  \item o estado atual do \textit{node};
  \item em qual estado o \textit{node} estava após a última execução do \textit{chef-client};
  \item em qual estado o \textit{node} deve alcançar ao final na execução do \textit{chef-client} atual.
\end{itemize}

Os \textit{attributes} são definidos: no proprio ambiente através do Ohai, nos \textit{cookbooks},
nos \textit{roles} e \textit{environments}.

\textbf{Cupper}:

\subsection{\textit{Recipes}}
\label{sec:lev-rec-rec}

Os \textit{recipes} são as unidade fundamentais para a execução de um \textit{cookbook}. Definem
todas as informações necessárias para configurar o sistema e segue algumas regras
~\cite{chefdoc:2016}:

\begin{itemize}
  \item Deve ser incluido em um \textit{cookbook};
  \item Deve ser posto em um \textit{run-list} antes de ser usado pelo \textit{chef-client};
  \item É executado na mesma ordem disposta na \textit{run-list};
  \item Pode ser incluído em outro \textit{recipe};
  \item Pode ter dependência de outros \textit{recipe};
  \item Pode marcar um \textit{node} para facilitar a criação de agrupamento.
\end{itemize}

As \textit{recipes} são escritas em Ruby e pode-se utilizar dos recursos providos
pela linguagem. Dispoem-se de uma coleção desse recursos que são utilizados
para definir as ações da \textit{recipes}. No exemplo \ref{code:recipe} utiliza-se
três recursos: \textit{package, service} e \textit{template}.

\begin{minipage}{.90\textwidth}
  \lstset{style=shell}
  \lstinputlisting[language=Bash, label=code:recipe, caption="Exemplo de \textit{recipe}. Define a instala{\c{c}}{\~a}o e configura{\c{c}}{\~a}o do \textit{app} Nginx"]{conteudo/code/recipe_example.rb}
\end{minipage}

\textbf{Cupper}:

\subsection{\textit{Definitions}}
\label{sec:lev-rec-def}

As \textit{definitions} são um novo tipo de recurso disponível apartir da versão
12.5 do Chef e é recomentado utilizar o \textit{Custom Resource}(seção \ref{sec:lev-rec-cust})
no lugar de \textit{definitions}~\cite{chefdoc:2016}.

Os \textit{definitions} são comportamento que podem ser reutilizados por outros \textit{recipes}.
São utilizado como recursos padrões de uma receita. No exemplo \ref{code:definition}
é definido o recurso \textit{host\_porter} com os parametros \textit{port} (valor padrão 4000)
e \textit{hostname} (valor padrão \textit{nil}) que pode ser utilizado em outro
\textit{recipe} com a simples chamada \textit{host\_porter}.

\begin{minipage}{.90\textwidth}
  \lstset{style=shell}
  \lstinputlisting[language=Bash, label=code:definition, caption="Exemplo de \textit{definition}. Adiciona um recurso \textit{host\_porter}."]{conteudo/code/definition_example.rb}
\end{minipage}

\textbf{Cupper}:

\subsection{\textit{Files}}

\textbf{Cupper}:

\subsection{\textit{Libraries}}

\textbf{Cupper}:

\subsection{\textit{Custom Resource}}
\label{sec:lev-rec-cust}

\textbf{Cupper}:

\subsection{\textit{Metadata}}

\textbf{Cupper}:

\subsection{\textit{Resources e Providers}}

\textbf{Cupper}:

\subsection{\textit{Templates}}

\textbf{Cupper}:

\subsection{\textit{Cookbook Version}}

\textbf{Cupper}:

\section{Relação entre Camadas e \textit{Cookbooks}}


