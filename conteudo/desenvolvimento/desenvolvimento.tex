\newpage\null\thispagestyle{empty}\newpage
\chapter{Desenvolvimento}
\label{chap:dev}

Esse capítulo irá descrever o desenvolvimento e evolução de cada uma das funcionalidades
previstas para o Cupper. Além disso, problemas encontrados durante o processo
de desenvolvimento serão relatados, e mudanças explicadas.

A maior e mais importante funcionalidade é a de gerar a receita Chef propriamente dita.
Dessa forma quebramos essa grande funcionalidade em funcionalidades menores que refletem cada
um dos tipos de extração e reprodução de recursos Chef.

As funcionalidades previstas para o Cupper são as seguintes:

\begin{enumerate}
  \item \textbf{Extract Packages}: Extrair atributos referentes aos pacotes
instalados e gerar receita Chef que instale os mesmos pacotes em um novo ambiente;
  \item \textbf{Extract Links}: Extrair atributos referentes a links simbólicos
de arquivos e gerar receita Chef que reproduza os mesmos links em um ambiente novo;
  \item \textbf{Extract Users}: Extrair atributos referentes a usuários presentes
no ambiente sendo extraído e gerar receita que crie os mesmos usuários em um
ambiente novo.
  \item \textbf{Extract Cookbook Files}: Extrair atributos referentes aos arquivos
de configuração padrão e gerar receita Chef que copie esses mesmos arquivos para o ambiente novo;
  \item \textbf{Extract Groups}: Extrair atributos referentes aos grupos de usuários
e gerar receita Chef que crie e associe os usuários certos a esses grupos em um
ambiente novo;
  \item \textbf{Extract Services}: Extrair atributos referentes aos serviços carregados
e ativos e gerar uma receita Chef que suba os mesmos serviços;
  \item \textbf{Gerar Projeto cupper}: Gerar a estrutura de diretório padrão 
para o projeto Cupper.
  \item \textbf{Listar Plugins}: Listar os plugins disponíveis do Ohai, tanto 
os padrões, já inclusos na instalação do Ohai, quanto os customizados pelo Cupper;
  \item \textbf{Help}: Imprime no terminal os principais comandos e as opções
válidas de cada uma;
  \item \textbf{Relatório Intermediário}: Gerar relatório intermediário antes
da receita Chef em si, para diagnóstico do sistema sendo extraído e para debug
do cupper. \textbf{Essa funcionalidade não foi implementada.}

\end{enumerate}

\section{Funcionalidade de extrair pacotes}
\label{sec:pacotes}

\subsection{Atributos}
Para a ter a funcionalidade mínima esperada desse recurso, definimos que iríamos
precisar extrair os seguintes atributos:

\begin{itemize}
  \item \textit{version}: especifica a versão do pacote;
  \item \textit{action}: define a ação que deve ser tomada
    (instalar, remover, reconfigurar, etc);
\end{itemize}

\subsection{\textit{Plugin} de Extração}

O Ohai provê um plugin nativo que coleta todos os pacotes instalados. O suporte
das plataformas são: Debian, Redhat, Fedora e OpenSUSE\@. Os dados coletados pelo
plugin de cada pacote são: nome do pacote e versão.

Para a plataforma Debian, o plugin utiliza a ferramenta \texttt{dpkg-query} disponível
por padrão nas distro Debian \textit{based}. O \texttt{dpkg-query} retorna uma lista de todos os pacotes
instalados no ambiente que é interpretada pelo plugin e transformada em um JSON
construído com a estrutura \texttt{``package\_name'' => \{ ``version'' => ``version\_number'' \}}
como mostrada no Código~\ref{code:json_pkg}.

\noindent\begin{minipage}{\textwidth}
  \lstset{style=shell}
  \lstinputlisting[frame=single,
    label=code:json_pkg,
    caption="Saída JSON do plugin Ohai \textit{packages}"]{conteudo/code/json_pkg.json}
\end{minipage}\hfill

Para as plataformas Redhat, Fedora e OpenSUSE, o plugin utiliza a ferramenta \texttt{rpm}
disponível por padrão nas distros Redhat \textit{based}. O Cupper ainda não oferece suporte
para distros Redhat \textit{based}, portanto não foram feitos testes para esse tipo de ambiente.

Não há suporte para a plataforma Arch Linux, portanto um plugin foi desenvolvido
para extrair as mesmas informações. A extração utiliza a ferramenta \texttt{pacman} disponível
nas distros Arch Linux. O \texttt{pacman} retorna uma lista de pacotes instalados no ambiente
que é interpretada pelo plugin e transformada em um JSON construído com a estrutura
\texttt{"package\_name" => \{ "version" => "version\_number" \}} como mostrada no
Código~\ref{code:json_pkg_pacman}.

\noindent\begin{minipage}{\textwidth}
  \lstset{style=shell}
  \lstinputlisting[frame=single,
    label=code:json_pkg_pacman,
    caption="Saída JSON do plugin Ohai \textit(pacman)"]{conteudo/code/json_pkg_pacman.json}
\end{minipage}\hfill

\subsection{Bloco de Receita Gerado}

O JSON gerado pela extração é utilizado para construir a receita responsável pela instalação dos
pacotes. É utilizado o recurso Chef \textit{package} para instalação com os dois atributos
\textit{version} e em \textit{action} (Código~\ref{code:pkg_recipe}). \textit{Version}
é extraído de cada pacote e \textit{action} é colocado \textit{\textit{:install}} como padrão para todos os pacotes.

\noindent\begin{minipage}{\textwidth}
  \lstset{style=shell}
  \lstinputlisting[
    language=Ruby,
    frame=single,
    label=code:pkg_recipe,
    caption="Exemplo de receita gerada pela extração de pacotes."]{conteudo/code/pkg_recipe.rb}
\end{minipage}\hfill

Considera-se que o pacote irá cuidar da suas próprias dependências quanto a instalação.
Sendo assim, os pacotes que são dependências de outros não são inclusos na receita. Exemplo, se o pacote
\textbf{A} contém as dependências \textbf{B} e \textbf{C} elas não serão inclusas na receita, pois o pacote \textbf{A} irá realizar a instalação.

\subsection{Problemas Encontrados}

\subsubsection{Pacotes com repositório fora do \textit{source list} para o 
gerenciador de pacotes}
Esse problema surgia quando um pacote que foi instalado manualmente, ou que
teve seu reposotório removido do \textit{source list} era extraído usando o Cupper,
mas ao executar a receita em um novo ambiente, a instalação falhava, já que 
o gerenciador de pacotes não encontrava a fonte do pacote.

Já que isso não é o comportamento esperado, mas pode ser uma escolha do usuário
ter pacotes de outras fontes, foi criado um atributo para que o usuário defina no
Cupperfile e decida se quer que se extraia pacotes que tem fonte em repositórios
fora do \textit{source list}. O exemplo de Cupperfile que é criado pelo comando
\textit{cupper create project} tem exemplos desses atributos e explicação, como
no Código~\ref{code:sourcelesspackages.rb}.

\noindent\begin{minipage}{0.7\textwidth}
  \lstset{style=shell}
  \lstinputlisting[language=Ruby, label=code:sourcelesspackages.rb, caption="Definição do atributo de \textit{sourceless packages}"]{conteudo/code/sourcelesspackages.rb}
\end{minipage}\hfill

\subsubsection{Pacotes já instalados no ambiete novo com versões menores}
Os pacotes já instlados no ambiente novo em que a receita vai rodar podem ter versões
maiores do que a dos pacotes extraídos do ambiente original. Se a receita for
montada sem nenhum tipo de tratamento, a sua execução vai quebrar no momento
em que ela tentar fazer esse \textit{downgrade}.

Foi criado então um atributo para o Cupperfile que deixa o usuário definir
se o Cupper vai forçar os \textit{downgrades} ou se vai ignorar os recursos
Cheff de pacotes que caírem nessa situação, passando \textit{action :nothing}
para esses. O exemplo vindo do Cupperfile com sua explicação pode ser visto no
Código~\ref{code:allowdowngrade}

\noindent\begin{minipage}{0.7\textwidth}
  \lstset{style=shell}
  \lstinputlisting[language=Ruby, label=code:allowdowngrade, caption="Definição do atributo de \textit{allow downgrade}"]{conteudo/code/allowdowngrade.rb}
\end{minipage}\hfill

\section{Funcionalidade de extrair links}
\label{sec:links}


\section{Funcionalidade de extrair usuários}
\label{sec:users}

\subsection{Atributos}

Para a ter a funcionalidade mínima esperada desse recurso, definimos que iríamos precisar
extrair os seguintes atributos:


\begin{itemize}
    {\itshape\item name}: Identifica nome do usuário;
    {\itshape\item uid}: Identificador do usuário;
    {\itshape\item home}: Identificar diretório \textit{home} desse usuário;
    {\itshape\item shell}: Identificar qual \textit{shell} o usuário usa ;
    {\itshape\item manage\_home}: Identificar se a \textit{home} deve ser criada;
\end{itemize}

\subsection{\textit{Plugin} de Extração}

\section{Funcionalidade de extrair arquivos de configuração}
\label{sec:files}
\subsection{Atributos}
\subsection{\textit{Plugin} de Extração}
\subsection{Bloco de Receita Gerado}
\subsection{Problemas Encontrados}

\subsubsection{Arquivos de configurações sensíveis com potencial de qubrar o sistema}
Todos os arquivos de configuração do \textit{/etc} estavam sendo copiados para
o novo sistema via receita. Isso incluia arquivo de \textit{sudoers}, \textit{/etc/group}
entre outros que tem potencial de quebrar o acesso ao sistema caso ocorra algum erro.

Para evitar esse problema, mas deixar também como opção para o usuário, mais um
atributo para o Cupperfile foi criado, definindo se arquivos sensíveis vão ser
copiados ou não. O exemplo vindo do Cupperfile com sua explicação pode ser visto no
Código~\ref{code:sensiblefiles}

\noindent\begin{minipage}{0.7\textwidth}
  \lstset{style=shell}
  \lstinputlisting[language=Ruby, label=code:sensiblefiles, caption="Definição do atributo de \textit{sensible files}"]{conteudo/code/sensiblefiles.rb}
\end{minipage}\hfill

\section{Funcionalidade de extrair grupos}
\label{sec:groups}


\section{Funcionalidade de extrair serviçoes}
\label{sec:services}

\subsection{Atributos}

O recurso de serviços do Chef prevê os seguintes atributos, alguns obrigatórios, outros não:

\begin{itemize}
{\itshape\item init\underline{ }command}
{\itshape\item pattern}
{\itshape\item priority}
{\itshape\item provider}
{\itshape\item reload\underline{ }command}
{\itshape\item restart\underline{ }command}
{\itshape\item service\underline{ }name}
{\itshape\item start\underline{ }command}
{\itshape\item stop\underline{ }command}
{\itshape\item subscribes}
{\itshape\item supports}
{\itshape\item timeout}
{\itshape\item action}
\end{itemize}

Para a ter a funcionalidade mínima esperada desse recurso, definimos que iríamos precisar
extrair os seguintes atributos:


\begin{itemize}
    {\itshape\item provider}: Identificar qual \textit{initSystem} está gerenciando
esse serviço;
    {\itshape\item service\underline{ }name}: Identificar o nome do serviço a
ser levantado;
    {\itshape\item action}: Identificar ação, padrão é restart;
\end{itemize}

\subsection{Plugin de Extração}

\subsection{Geração de bloco de receita}

\subsection{Problemas encontrados}

issues

\subsubsection{Resultados com problemas}

\subsection{Mudanças e Justificativas}

\subsection{Resultados da funcionalidade de extrair serviçoes}

\section{Funcionalidade gerar diretório de projeto}
\label{sec:proj}

Essa funcionalidade cria a estrutura de diretório para que o Cupper gere as 
receitas Chef e leia o seu arquivo de configuração, o Cupperfile.

Como exibido no Código~\ref{code:thorsaida2}, o básico dessa funcionalidade é, bem
simples. Um diretório com o nome escolhido vai ser criado, dentro dele um diretório
para as cookbooks, um arquivo oculto para listar arquivos sensíveis, e um 
Cupperfile para as definições que o usuário define para a extração.

\subsection{Definição de ambiente}
Existem algumas classes importantes no código do Cupper que são relacionadas a
definição do ambiente de execução. Umas deas é a classe do \textit{Cupperfile} e outra
é o \textit{Environment}.

A classe de \textit{Environment} é responsável por checar se o ambiente é um
ambiente válido com a presença de um Cupperfile, definir o diretório do ambiente
(\textit{root\underline{ }path}) e instanciar um \textit{Cupperfile}. A classe \textit{Cupperfile} carrega as configurações definidas no arquivo Cupperfile.

\subsection{Problemas encontrados}
Devido a priorização realizada, e com o andamento do projeto, o carregamento de
configurações adicionais vindas do Cupperfile tinha sido tirado do backlog do
trabalho.

Essa decisão seguiu por algum tempo até que algumas falhas fizeram o Cupperfile
ser indispensável para deixar o Cupper funcionando para mais casos.

\subsubsection{Pacotes com repositório fora do \textit{source list} para o 
gerenciador de pacotes}
Esse problema surgia quando um pacote que foi instalado manualmente, ou que
teve seu reposotório removido do \textit{source list} era extraído usando o Cupper,
mas ao executar a receita em um novo ambiente, a instalação falhava, já que 
o gerenciador de pacotes não encontrava a fonte do pacote.

Já que isso não é o comportamento esperado, mas pode ser uma escolha do usuário
ter pacotes de outras fontes, foi criado um atributo para que o usuário defina no
Cupperfile e decida se quer que se extraia pacotes que tem fonte em repositórios
fora do \textit{source list}. O exemplo de Cupperfile que é criado pelo comando
\textit{cupper create project} tem exemplos desses atributos e explicação, como
no Código~\ref{code:sourcelesspackages.rb}.

\noindent\begin{minipage}{0.7\textwidth}
  \lstset{style=shell}
  \lstinputlisting[language=Ruby, label=code:sourcelesspackages.rb, caption="Definição do atributo de \textit{sourceless packages}"]{conteudo/code/sourcelesspackages.rb}
\end{minipage}\hfill

\subsubsection{Pacotes já instalados no ambiete novo com versões menores}
Os pacotes já instlados no ambiente novo em que a receita vai rodar podem ter versões
maiores do que a dos pacotes extraídos do ambiente original. Se a receita for
montada sem nenhum tipo de tratamento, a sua execução vai quebrar no momento
em que ela tentar fazer esse \textit{downgrade}.

Foi criado então um atributo para o Cupperfile que deixa o usuário definir
se o Cupper vai forçar os \textit{downgrades} ou se vai ignorar os recursos
Cheff de pacotes que caírem nessa situação, passando \textit{action :nothing}
para esses. O exemplo vindo do Cupperfile com sua explicação pode ser visto no
Código~\ref{code:allowdowngrade}

\noindent\begin{minipage}{0.7\textwidth}
  \lstset{style=shell}
  \lstinputlisting[language=Ruby, label=code:allowdowngrade, caption="Definição do atributo de \textit{allow downgrade}"]{conteudo/code/allowdowngrade.rb}
\end{minipage}\hfill

\subsubsection{Arquivos de configurações sensíveis com potencial de qubrar o sistema}
Todos os arquivos de configuração do \textit{/etc} estavam sendo copiados para
o novo sistema via receita. Isso incluia arquivo de \textit{sudoers}, \textit{/etc/group}
entre outros que tem potencial de quebrar o acesso ao sistema caso ocorra algum erro.

Para evitar esse problema, mas deixar também como opção para o usuário, mais um
atributo para o Cupperfile foi criado, definindo se arquivos sensíveis vão ser
copiados ou não. O exemplo vindo do Cupperfile com sua explicação pode ser visto no
Código~\ref{code:sensiblefiles}

\noindent\begin{minipage}{0.7\textwidth}
  \lstset{style=shell}
  \lstinputlisting[language=Ruby, label=code:sensiblefiles, caption="Definição do atributo de \textit{sensible files}"]{conteudo/code/sensiblefiles.rb}
\end{minipage}\hfill

\subsubsection{Outras abordagens de leitura do Cupperfile}
Para a leitura e carregamento das configurações definidas no Cupperfile diversas
abordagens foram consideradas e testadas. A primeira abordagem foi espelhar a
leitura do Cupperfile ao que o Vagrant faz com o Vagrantfile. O Vagrant é outra
ferramenta dentro do contexto de DevOps que inspirou algumas das decisões arquiteturais
do Cupper. Assim como o Vagrant, essa abordagem iria carregar as configurações
do Cupperfile que seria escrito em Ruby. O código do Cupperfile seria carregado
utilizando o \textit{Kernel.load}, que carrega e executa códigos externos.
Essa abordagem foi abandonada pois previa diversas instancias do \textit{loader}
e de objetos que possuiriam informações de vários ambientes (como é necessário
no Vagrant), e isso se mostrou desnecessário e muito complexo para o Cupper.

A segunda abordagem para o carregamento das configurações do Cupperfile foi a de
utilizar uma Gem externa (\textit{onfiguration}) que define um padrão de escrita de 
arquivos de configuração (semelhante a um JSON), e permite o carregamento 
simples deles. Essa também foi abandonada por adicionar mais dependencias para o Cupper.

A terceira e quarta abordagens voltaram a tentar utilizar Ruby puro para resolver
esse processo de carregamento de arquivos de configuração, dessa vez tentando 
fazer um módulo mais simples que prevesse somente um ambiente em que o Cupper
iria operar. A quarta abordagem consistiu de somente tornar isso global e
pertencente do módulo \textit{Cupper::Config} acessível em todo o código.

\textit{Links} úteis para carregamento de arquivos de configuração de todas essas
abordagens podem ser encontrados no Apêndice na seção~\ref{apc:conf-file}.

\subsection{Mudanças e Justificativas}
O escopo tinha sido alterado, e o carregamento de configurações do Cupperfile
tinha sido removido do backlog. Mas esse carregamento se mostrou necessário
e acabou voltando para o desenvolvimento.

Algumas dessas definições do Cupperfile foram reativas, e implementadas de acordo
com o surgimento da necessidade de implementar. Outras tinham sido previstas, mas
não prioritárias, como a de \textit{allow downgrade}.

O que fez essa abordagem voltar a ser indispensável foi o fato de que o Cupper
não pode gerar uma receita que quebra o acesso do usuário a um sistema, ou pior,
que inutilize o sistema.

\section{Funcionalidade de Listar \textit{Plugins}}
\label{sec:list}


\section{Funcionalidade de Ajuda}
\label{sec:help}

\section{Funcionalidade gerar relatório intermediário}
\label{sec:rel}

