\newpage\null\thispagestyle{empty}\newpage
\chapter{Desenvolvimento}
\label{chap:dev}

Esse capítulo irá descrever o desenvolvimento e evolução de cada uma das funcionalidades
previstas para o Cupper. Além disso, problemas encontrados durante o processo
de desenvolvimento serão relatados, e mudanças explicadas.

A maior e mais importante funcionalidade é a de gerar a receita Chef propriamente dita.
Dessa forma quebramos essa grande funcionalidade em funcionalidades menores que refletem cada
um dos tipos de extração e reprodução de recursos Chef.

As funcionalidades previstas para o Cupper são as seguintes:

\begin{enumerate}
  \item \textbf{Extract Packages}: Extrair atributos referentes aos pacotes
instalados e gerar receita Chef que instale os mesmos pacotes em um novo ambiente;
  \item \textbf{Extract Links}: Extrair atributos referentes a links simbólicos
de arquivos e gerar receita Chef que reproduza os mesmos links em um ambiente novo;
  \item \textbf{Extract Users}: Extrair atributos referentes a usuários presentes
no ambiente sendo extraído e gerar receita que crie os mesmos usuários em um
ambiente novo.
  \item \textbf{Extract Cookbook Files}: Extrair atributos referentes aos arquivos
de configuração padrão e gerar receita Chef que copie esses mesmos arquivos para o ambiente novo;
  \item \textbf{Extract Groups}: Extrair atributos referentes aos grupos de usuários
e gerar receita Chef que crie e associe os usuários certos a esses grupos em um
ambiente novo;
  \item \textbf{Extract Services}: Extrair atributos referentes aos serviços carregados
e ativos e gerar uma receita Chef que suba os mesmos serviços;
  \item \textbf{Gerar Projeto cupper}: Gerar a estrutura de diretório padrão 
para o projeto Cupper.
  \item \textbf{Listar Plugins}: Listar os plugins disponíveis do Ohai, tanto 
os padrões, já inclusos na instalação do Ohai, quanto os customizados pelo Cupper;
  \item \textbf{Help}: Imprime no terminal os principais comandos e as opções
válidas de cada uma;
  \item \textbf{Relatório Intermediário}: Gerar relatório intermediário antes
da receita Chef em si, para diagnóstico do sistema sendo extraído e para debug
do cupper. \textbf{Essa funcionalidade não foi implementada.}

\end{enumerate}

\section{Funcionalidade de Extrair Pacotes}
\label{sec:pacotes}

\subsection{Atributos}
Para ter a funcionalidade mínima esperada desse recurso, definimos que iríamos
precisar extrair os seguintes atributos:

\begin{itemize}
  \item \textit{version}: especifica a versão do pacote;
  \item \textit{action}: define a ação que deve ser tomada
    (instalar, remover, reconfigurar, etc);
\end{itemize}

\subsection{\textit{Plugin} de Extração}

O Ohai provê um plugin nativo que coleta todos os pacotes instalados. O suporte
das plataformas são: Debian, Redhat, Fedora e OpenSUSE\@. Os dados coletados pelo
plugin de cada pacote são: nome do pacote e versão.

Para a plataforma Debian, o plugin utiliza a ferramenta \texttt{dpkg-query} disponível
por padrão nas distro Debian \textit{based}. O \texttt{dpkg-query} retorna uma lista de todos os pacotes
instalados no ambiente que é interpretada pelo plugin e transformada em um JSON
construído com a estrutura \texttt{``package\_name'' => \{ ``version'' => ``version\_number'' \}}
como mostrada no Código~\ref{code:json_pkg}.

\noindent\begin{minipage}{\textwidth}
  \lstset{style=shell}
  \lstinputlisting[frame=single,
    label=code:json_pkg,
    caption="Saída JSON do plugin Ohai \textit{packages}"]{conteudo/code/json_pkg.json}
\end{minipage}\hfill

Para as plataformas Redhat, Fedora e OpenSUSE, o plugin utiliza a ferramenta \texttt{rpm}
disponível por padrão nas distros Redhat \textit{based}. O Cupper ainda não oferece suporte
para distros Redhat \textit{based}, portanto não foram feitos testes para esse tipo de ambiente.

Não há suporte para a plataforma Arch Linux, portanto um plugin foi desenvolvido
para extrair as mesmas informações. A extração utiliza a ferramenta \texttt{pacman} disponível
nas distros Arch Linux. O \texttt{pacman} retorna uma lista de pacotes instalados no ambiente
que é interpretada pelo plugin e transformada em um JSON construído com a estrutura
\texttt{"package\_name" => \{ "version" => "version\_number" \}} como mostrada no
Código~\ref{code:json_pkg_pacman}.

\noindent\begin{minipage}{\textwidth}
  \lstset{style=shell}
  \lstinputlisting[frame=single,
    label=code:json_pkg_pacman,
    caption="Saída JSON do plugin Ohai \textit(pacman)"]{conteudo/code/json_pkg_pacman.json}
\end{minipage}\hfill

\subsection{Bloco de Receita Gerado}

O JSON gerado pela extração é utilizado para construir a receita responsável pela instalação dos
pacotes. É utilizado o recurso Chef \textit{package} para instalação de cada pacote com os dois atributos
\textit{version} e \textit{action} (Código~\ref{code:pkg_recipe}). \textit{Action} é colocado \textit{\textit{:install}}
como padrão para todos os pacotes.

\noindent\begin{minipage}{\textwidth}
  \lstset{style=shell}
  \lstinputlisting[
    language=Ruby,
    frame=single,
    label=code:pkg_recipe,
    caption="Exemplo de receita gerada pela extração de pacotes."]{conteudo/code/pkg_recipe.rb}
\end{minipage}\hfill

Considera-se que o pacote irá cuidar da suas próprias dependências quanto a instalação.
Sendo assim, os pacotes que são dependências de outros não são inclusos na receita. Exemplo, se o pacote
\textbf{A} contém as dependências \textbf{B} e \textbf{C} elas não serão inclusas na receita, pois o pacote \textbf{A} irá realizar a instalação.

\subsection{Problemas Encontrados}

O método de extração dos pacotes instalados não realiza nenhuma checagem para
saber se o pacote continua disponível no repositório de pacotes remotos ou
se ao menos tem um repositório remoto associado. Isso causou falha na realização
dos testes. Tanto os pacotes instalados manualmente, por meio de download do
\textit{source}, quanto os pacotes que não estão mais disponíveis no repositório remoto ou
mesmo repositórios inseridos manualmente, não eram identificado durante a
configuração do novo ambiente.

Esse comportamento não é esperado, portanto algumas medidas foram tomadas para
que o problema fosse parcialmente corrigido. Para o caso de repositórios que
tenha outro \textit{source} além do padrão Debian, a receita de package gerada adiciona
o atributo \textit{cookbook\_file} contendo o arquivo \texttt{source.list} e outros disponíveis
no diretório \texttt{/etc/apt/source.list.d/} extraído do ambiente. Isso permite
que todos os repositórios não oficiais do ambiente alvo sejam aplicados ao novo ambiente.

% TODO: isso será trabalhos futuros
%Já que isso não é o comportamento esperado, mas pode ser uma escolha do usuário
%ter pacotes de outras fontes, foi criado um atributo para que o usuário defina no
%Cupperfile e decida se quer extrair pacotes que tem fonte em repositórios
%fora do \textit{source list}. O exemplo de Cupperfile que é criado pelo comando
%\textit{cupper create project} tem exemplos desses atributos e explicação, como
%no Código~\ref{code:sourcelesspackages.rb}.
%
%\noindent\begin{minipage}{\textwidth}
%  \lstset{style=shell}
%  \lstinputlisting[
%    language=Ruby,
%    label=code:sourcelesspackages.rb,
%  caption="Definição do atributo de \textit{sourceless packages}"]{conteudo/code/sourcelesspackages.rb}
%\end{minipage}\hfill

Outro problema encontrado está relacionado a versão dos pacotes.
A receita especifica para todos os pacotes a versão que foi extraída do ambiente.
Ao realizar a configuração do novo ambiente, é possível que o pacote já esteja instalado,
mas a sua versão esta à frente da definida na receita. Quando isso ocorre, dois comportamentos
podem ocorrer: não há error e é feito o \textit{downgrade} do pacote ou ocorre erro no \textit{downgrade} das
dependências do pacote. Isso ocorre porque o gerenciador de pacotes não realiza o \textit{downgrade}
das dependências, e as versões antigas dos pacotes podem exigir as versões antigas das suas
dependências.

%Portanto foi criado o atributo \texttt{allow\_downgrade} para o Cupperfile que possibilita
%o usuário definir se a receita gerada irá conter atributos que force instalar
%a versão antiga (Código~\ref{code:allowdowngrade}). O método de realizar isso
%é por meio da desinstalação completa do pacote, que inclui as dependências,
%e a instalação do pacote antigo.
%
%\noindent\begin{minipage}{\textwidth}
%  \lstset{style=shell}
%  \lstinputlisting[
%    language=Ruby,
%    label=code:allowdowngrade,
%    caption="Definição do atributo de \textit{allow downgrade}"]{conteudo/code/allowdowngrade.rb}
%\end{minipage}\hfill

Ainda com relação a versão do pacote, outro caso de erro foi encontrado.
Por conta do sistema estar desatualizado, ou seja, não houve atualização dos pacotes
instalados, a versão instalada pode não estar mais disponível no repositório principal.
Neste caso o gerenciado de pacotes não encontrará a fonte e acusará erro.


\section{Funcionalidade de extrair arquivos de configuração}
\label{sec:files}

\section{Funcionalidade de Extrair Links}
\label{sec:links}

\subsection{Atributos}
Para o recurso Chef de \texttt{link} foi definido os seguintes
atributos para a extração:

\begin{itemize}
  \item \textit{group}: define o grupo que pertence o link;
  \item \textit{owner}: dono do link, referente ao usuário no ambiente;
  \item \textit{mode}: permissões do link;
  \item \textit{to}: caminho para o arquivo real;
\end{itemize}

\subsection{\textit{Plugin} de Extração}

O mesmo plugin utilizado para recuperar arquivos de configuração descritos
na Seção~\ref{sec:files} também é usado para extrair \textit{links}. Isso é possível
pelo atributo \textit{type} coletado pelo plugin. Nele está contido, além do tipo do arquivo,
o caminho para o arquivo real. Os outros atributos são comuns entre arquivos e
\textit{links}, que são coletados e tratados da mesma forma. O JSON gerado segue a
mesma estrutura dos arquivos como mostra o Código~\ref{code:json_links}.

\noindent\begin{minipage}{\textwidth}
  \lstset{style=shell}
  \lstinputlisting[
    label=code:json_links,
    caption="Saída JSON do plugin Ohai \textit{files} com exemplo de link."]{conteudo/code/json_links.json}
\end{minipage}\hfill

\subsection{Bloco de Receita Gerado}

A receita gerada usa o recurso \texttt{links} (Código~\ref{code:links_recipe}) com os
atributos para cada \textit{link} gerado.

\noindent\begin{minipage}{\textwidth}
  \lstset{style=shell}
  \lstinputlisting[
    label=code:links_recipe,
    caption="Exemplo de receita gerada pela extração de links."]{conteudo/code/links_recipe.rb}
\end{minipage}\hfill

O atributo \textit{mode} é irrelevante neste caso. Todos os \textit{links} gerados tem a mesma permissão
777, pois o que determina a leitura, escrita ou execução do \textit{link} é o arquivo para o qual ele
direciona. Sendo assim, esse atributo é irrelevante e será removido em atualizações futuras.

\subsection{Problemas Encontrados}

Durante os testes de configuração dos novos ambientes, os \textit{links} continham caracteres
especias que não eram reconhecidos pelo \textit{encode} padrão do Ruby causando erro.
Esse tipo de texto requer um tratamento especial dentro do código, mas as tentativas
de correção desse problema não tiveram resultados positivos. A alternativa que encontramos
para contornar temporariamente esse problema é a remoção de qualquer link ou arquivo que
não estão nos formatos de \textit{encode} padrão suportados pela ferramenta.


\section{Funcionalidade de extrair usuários}
\label{sec:users}

\subsection{Atributos}

Para a ter a funcionalidade mínima esperada desse recurso, definimos que iríamos precisar
extrair os seguintes atributos:


\begin{itemize}
    {\itshape\item name}: Identifica nome do usuário;
    {\itshape\item uid}: Identificador do usuário;
    {\itshape\item home}: Identificar diretório \textit{home} desse usuário;
    {\itshape\item shell}: Identificar qual \textit{shell} o usuário usa ;
    {\itshape\item manage\_home}: Identificar se a \textit{home} deve ser criada;
\end{itemize}

\subsection{\textit{Plugin} de Extração}
Não foi necessário desenvolver um \textit{plugin} para extrair usuários pois o Ohai
já possui um \textit{plugin} nativo para isso. Ele provê atributos referentes
a informações de usuários, grupos, permissões entre outras.

Esse \textit{plugin} utiliza o módulo Ruby \textit{etc} que provê acesso a informações
do sistema operacional, diferentemente de alguns outros \textit{plugins} do Ohai
que executam comandos \textit{shell} diretamente na máquina. A saída é interpretada
pelo \textit{plugin} e transformada em um JSON construído com a estrutura
\texttt{``etc''=> \{ ``passwd''=> \{ ``root''=> \{ ``dir''=> ``/root'', ``gid''=> 0, ``uid''=> 0, ``shell''=> ``/bin/bash'', ``gecos''=> ``root'' \}} 
como mostrada no Código~\ref{code:json_user}

\noindent\begin{minipage}{\textwidth}
  \lstset{style=shell}
  \lstinputlisting[frame=single,
    label=code:json_user,
    caption="Saída JSON do plugin Ohai \textit{passwd} para usuários"]{conteudo/code/json_user.json}
\end{minipage}\hfill

\subsection{Bloco de Receita Gerado}


\section{Funcionalidade de extrair grupos}
\label{sec:groups}

\subsection{Atributos}

Para a ter a funcionalidade mínima esperada desse recurso, definimos que iríamos precisar
extrair os seguintes atributos:

\begin{itemize}
    {\itshape\item gid}: Identificador do grupo;
    {\itshape\item members}: Identificar membros do grupo;
\end{itemize}


\begin{itemize}
    {\itshape\item append}: Definir que os membros podem ser adicionados e removidos;
    {\itshape\item action}: Definir ação de criar o grupo;
\end{itemize}

\subsection{\textit{Plugin} de Extração}
Não foi necessário criar um \textit{plugin} para o Ohai que extraísse esses atributos
referentes à grupos já que o Ohai já possui um. O nome \textit{plugin} usado é 
\textit{passwd} e ele provê informações de grupos, usuários, permissões entre outras.

\subsection{Geração de bloco de receita}

Para a criação da receita que cria os grupos em um novo ambiente, foi definido
como seria o template dos blocos de recurso de grupo na sintaxe das receitas Chef.
Um exemplo pode ser visto no código~\ref{code:groupresource}.

\noindent\begin{minipage}{\textwidth}
  \lstset{style=shell}
  \lstinputlisting[language=Ruby, frame=single, label=code:groupresource, caption="Bloco de recurso de grupo Chef"]{conteudo/code/groupresource.rb}
\end{minipage}\hfill

Se uma receita com esse recurso for executada em um ambiente, o grupo \textit{www-data}
irá ser modificado com um novo membro \textit{maintenance}.

Para gerar então diversos desses blocos na receita final o template do código~\ref{code:grouptemplate} foi definido.

\noindent\begin{minipage}{\textwidth}
  \lstset{style=shell}
  \lstinputlisting[language=Ruby, frame=single, label=code:grouptemplate, caption="Template para os blocos de recursos de grupo Chef"]{conteudo/code/grouptemplate.rb}
\end{minipage}\hfill

Dessa forma, como visto acima, para criar um bloco de receita de recursos Chef é
necessário saber qual vai ser o template da escrita desse bloco e qual \textit{plugin}
o coletor vai utilizar para a extração.

\subsection{Problemas encontrados}

Essa funcionalidade de extração de grupos não tinha sido prevista inicialmente.
Como citado na Seção~\ref{sec:users}, ela nasceu de um problema com a criação de usuários
de outra funcionalidade. Sem a organização desses usuários em grupos, o Chef não conseguia
executar a receita.

% mapear quais problemas bota aqui e quais bota em users mesmo

\section{Funcionalidade de extrair serviçoes}
\label{sec:services}

\subsection{Atributos}

O recurso de serviços do Chef prevê os seguintes atributos, alguns obrigatórios, outros não:

\begin{itemize}
{\itshape\item init\underline{ }command}
{\itshape\item pattern}
{\itshape\item priority}
{\itshape\item provider}
{\itshape\item reload\underline{ }command}
{\itshape\item restart\underline{ }command}
{\itshape\item service\underline{ }name}
{\itshape\item start\underline{ }command}
{\itshape\item stop\underline{ }command}
{\itshape\item subscribes}
{\itshape\item supports}
{\itshape\item timeout}
{\itshape\item action}
\end{itemize}

Para a ter a funcionalidade mínima esperada desse recurso, definimos que iríamos precisar
extrair os seguintes atributos:


\begin{itemize}
    {\itshape\item provider}: Identificar qual \textit{Init System} está gerenciando
esse serviço;
    {\itshape\item service\underline{ }name}: Identificar o nome do serviço a
ser levantado;
    {\itshape\item action}: Identificar ação, padrão é restart;
\end{itemize}

\subsection{Plugin de Extração}
Para a extração desses atributos referentes aos serviços, foi criado um plugin do
Ohai que, a partir do comando \textit{systemctl} lista todos os serviços do ambiente
e extrai os atributos de \textit{service\underline{ }name}, \textit{provider}
e \textit{action}.

Esses atributos são coletados e para cada serviços extraído um objeto de serviço
é criado para ser iterado no template posteriormente.

\subsection{Geração de bloco de receita}
Para a geração da receita que levanta esses serviços em outro ambiente, foi 
definido como seria o template dos blocos de recurso de serviços na sintaxe 
\textit{Ruby like} própria das receitas Chef.

Um exemplo de bloco de recurso de serviço pode ser visto no código~\ref{code:serviceresource}.

\noindent\begin{minipage}{\textwidth}
  \lstinputlisting[language=Ruby, frame=single, label=code:serviceresource, caption="Bloco de recurso de serviço Chef"]{conteudo/code/serviceresource.rb}
\end{minipage}\hfill

Esse exemplo de bloco de recurso de serviço Chef, se executado em um ambiente, diz para o
gerenciador de serviços (\textit{Init System}) dar um start no serviço do tomcat.

Dessa forma, para gerar diversos desses blocos na receita final, temos o template
do código~\ref{code:servicetemplate}.

\noindent\begin{minipage}{\textwidth}
  \lstinputlisting[language=Ruby, frame=single, label=code:servicetemplate, caption="Template para os blocos de recursos de serviço Chef"]{conteudo/code/servicetemplate.rb}
\end{minipage}\hfill

Então para gerar um bloco de receita de recursos de serviço Chef é necessário
saber como vai ser o template da escrita desse bloco e qual \textit{plugin} o
coletor vai utilizar para a extração.

No código~\ref{code:servicerecipe} é possível ver a chamada para a inicialização e escrita de um
bloco de receita de serviços. Todos os parâmetros necessários são passados para
o construtor da classe e então o método create que escreve a receita é chamado.
O parâmetro \textit{collector} passa um coletor genérico que carregará todos os
atributos desse e de outros recursos, o parâmetro \textit{`\underline{ }services'} é uma
string com o nome do template a ser usado para escrever a receita, e o ultimo
parâmetro é uma string que indica qual \textit{plugin} o coletor deverá utilizar.

\noindent\begin{minipage}{\textwidth}
  \lstinputlisting[language=Ruby, frame=single, label=code:servicerecipe, caption="Chamada para inicialização e escrita de um bloco de receita"]{conteudo/code/servicerecipe.rb}
\end{minipage}\hfill



\subsection{Problemas encontrados}

\subsubsection{Distinção entre serviços instalados e serviços rodando}
A primeira abordagem utilizada para a extração dos atributos referentes a esse
recurso não previa a distinção entre serviços instalados e serviços que estão
ativos ou estão rodando. A extração ocorria sem nenhum problema, e todos os serviços
instalados eram extraídos. Mas ao executar a receita em um novo ambiente, diversos
serviços falhavam ao tentar se inicializar.

Serviços que já estavam falhando ou já estavam inativos no ambiente sendo extraído
também iriam falhar no ambiente novo a ser gerado, e isso teve que ser tratado.


Para solucionar esse problema só serviços que tem status \textit{loaded, active e running}
passaram a ser extraídos.\

\subsubsection{Distinção entre diferentes \textit{Init Systems}}
Outro problema similar foi o de não verificar qual Init System gerenciava cada
serviço. Já que o recurso de serviço do Chef considera por padrão que os serviços
são \textit{systemd}, quando algum serviço não era, falhas aconteciam.

A extração passou então a extrair o atributo \textit{provider} que ao ser passado
para o recurso de serviço do Chef, solucionava esse problema.

\subsubsection{Distinção entre \textit{systemd units}}
O comando utilizado para que o \textit{plugin} do Ohai extraia os serviços não
lista somente serviços em si, já que ele lista todo tipo de \textit{systemd units}
como \textit{mounts, devices, targets, paths}. Esse comportamento não é ideal
tanto por gerar falhas quanto por não fazer parte do escopo interagir com esses
outros tipos de \textit{systemd units}.

O \textit{plugin} passou então a parsear as unidades e só pegar os que são serviços.

\subsection{Mudanças e Justificativas}
Nenhuma mudança com relação ao que era previsto para a funcionalidade de extrair
atributos relacionados a serviços foi feita. As mudanças que foram feitas com relação
às abordagens de desenvolvimento se devem somente ao fato de fazer a
funcionalidade ter o comportamento esperado.

\section{Funcionalidade gerar diretório de projeto}
\label{sec:proj}

Essa funcionalidade cria a estrutura de diretório para que o Cupper gere as 
receitas Chef e leia o seu arquivo de configuração, o Cupperfile.

Como exibido no Código~\ref{code:thorsaida2}, o básico dessa funcionalidade é, bem
simples. Um diretório com o nome escolhido vai ser criado, dentro dele um diretório
para as cookbooks, um arquivo oculto para listar arquivos sensíveis, e um 
Cupperfile para as definições que o usuário define para a extração.

\subsection{Definição de ambiente}
Existem algumas classes importantes no código do Cupper que são relacionadas a
definição do ambiente de execução. Umas deas é a classe do \textit{Cupperfile} e outra
é o \textit{Environment}.

A classe de \textit{Environment} é responsável por checar se o ambiente é um
ambiente válido com a presença de um Cupperfile, definir o diretório do ambiente
(\textit{root\underline{ }path}) e instanciar um \textit{Cupperfile}. A classe \textit{Cupperfile} carrega as configurações definidas no arquivo Cupperfile.

\subsection{Problemas encontrados}
Devido a priorização realizada, e com o andamento do projeto, o carregamento de
configurações adicionais vindas do Cupperfile tinha sido tirado do backlog do
trabalho.

Essa decisão seguiu por algum tempo até que algumas falhas fizeram o Cupperfile
ser indispensável para deixar o Cupper funcionando para mais casos.

\subsubsection{Outras abordagens de leitura do Cupperfile}
Para a leitura e carregamento das configurações definidas no Cupperfile diversas
abordagens foram consideradas e testadas. A primeira abordagem foi espelhar a
leitura do Cupperfile ao que o Vagrant faz com o Vagrantfile. O Vagrant é outra
ferramenta dentro do contexto de DevOps que inspirou algumas das decisões arquiteturais
do Cupper. Assim como o Vagrant, essa abordagem iria carregar as configurações
do Cupperfile que seria escrito em Ruby. O código do Cupperfile seria carregado
utilizando o \textit{Kernel.load}, que carrega e executa códigos externos.
Essa abordagem foi abandonada pois previa diversas instancias do \textit{loader}
e de objetos que possuiriam informações de vários ambientes (como é necessário
no Vagrant), e isso se mostrou desnecessário e muito complexo para o Cupper.

A segunda abordagem para o carregamento das configurações do Cupperfile foi a de
utilizar uma Gem externa (\textit{onfiguration}) que define um padrão de escrita de 
arquivos de configuração (semelhante a um JSON), e permite o carregamento 
simples deles. Essa também foi abandonada por adicionar mais dependencias para o Cupper.

A terceira e quarta abordagens voltaram a tentar utilizar Ruby puro para resolver
esse processo de carregamento de arquivos de configuração, dessa vez tentando 
fazer um módulo mais simples que prevesse somente um ambiente em que o Cupper
iria operar. A quarta abordagem consistiu de somente tornar isso global e
pertencente do módulo \textit{Cupper::Config} acessível em todo o código.

\textit{Links} úteis para carregamento de arquivos de configuração de todas essas
abordagens podem ser encontrados no Apêndice na Seção~\ref{apc:conf-file}.

\subsection{Mudanças e Justificativas}
O escopo tinha sido alterado, e o carregamento de configurações do Cupperfile
tinha sido removido do backlog. Mas esse carregamento se mostrou necessário
e acabou voltando para o desenvolvimento.

Algumas dessas definições do Cupperfile foram reativas, e implementadas de acordo
com o surgimento da necessidade de implementar. Outras tinham sido previstas, mas
não prioritárias, como a de \textit{allow downgrade}.

O que fez essa abordagem voltar a ser indispensável foi o fato de que o Cupper
não pode gerar uma receita que quebra o acesso do usuário a um sistema, ou pior,
que inutilize o sistema.

\section{Funcionalidade de Listar \textit{Plugins}}
\label{sec:list}

Listar \textit{plugins} é uma funcionalidade de suporte para o usuário
que mostra quais os plugins adicionais do Ohais que estão
disponíveis. É importante realizar a verificação, pois
sem a disponibilidade deles não é possível realizar a extração
dos atributos do ambiente.

\section{Funcionalidade de Ajuda}
\label{sec:help}

\section{Funcionalidade gerar relatório intermediário}
\label{sec:rel}

