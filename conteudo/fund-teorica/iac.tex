\section{Infraestrutura como Código}

Segundo ~\citeonline{huttermann:2012}, em linhas gerais, o que é considerado
infraestrutura são os itens como sistema operacional, servidoes,
\textit{switches} e \textit{routers}, mas pode, também, ser a combinação
de todos os ambientes da empresa e os serviços de suporte (\textit{firewall},
sistema de monitoramento, etc).

Antes dos conceitos de DevOps e dos movimentos ágeis, as configurações de
infraestrutura eram automatizadas com \textit{scripts} que geralmente eram
difíceis de serem compreendidos por alguém que não fosse o autor~\cite{huttermann:2012}.
Recentemente, o termo Infraestrutura como Código veem se popularizando
seguindo a mesma lógica que era informalmente aplicada anteriormente, criando
\textit{scripts} de automação de configuração para a infraestrutura.

A infraestrutura como código é focada em manipular a configuração da infraestrutura
da mesma maneira que os desenvolvedores manipulam os seus códigos: escolhendo a melhor
linguagem e ferramenta para o desenvolvimentos da solução, transformando uma especificação
em algo executável que possa ser aplicada em um sistema de forma eficiente e
repetível~\cite{huttermann:2012}. \\

\noindent\begin{minipage}{.45\textwidth}
  \label{code:shell}
  \lstset{style=shell}
  \lstinputlisting[language=Bash, label=code:shell, caption="Código em Shell"]{conteudo/code/shell_example.sh}
\end{minipage}\hfill
\begin{minipage}{.45\textwidth}
  \label{code:chef}
  \lstset{style=shell}
  \lstinputlisting[language=Bash, label=code:chef, caption="Código em Chef"]{conteudo/code/chef_example.rb}
\end{minipage}


Neste trabalho será abordado a ferramenta Chef, que consiste em uma ferramenta de
automação de configuração de infraestrutura da qual utiliza o conceito de infraestrutura
como código para estabelecer os \textit{scripts} configuração. Na a representação \ref{code:shell}
\ref{code:chef} é um exemplo de um \textit{script} escrito em \textit{shell} e outro em \textit{Ruby}
interpretada pela ferramenta Chef.
