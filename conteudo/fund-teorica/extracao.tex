\section{Ferramentas de Extração de Configuração}

%TODO: procurar ferramentas de extração de configuração de ambiente especificar o que é uma ferramenta de extração de configuração. Cabe aqui ferramentas ja existentes.

\subsection{Blueprint}

O Blueprint é uma ferramenta de gerência de configuração que realiza a
engenharia reversa do sistema para extrair, em conjunto a outras ferramentas,
as informações de pacotes, serviços e fontes de instalação de aplicativos
~\cite{blueprint:2016}. O Blueprint foi descontinuado em 2014.

A aplicação é utilizada para criar \textit{scripts} que realizam
a instalação dos pacotes e configurações dos serviços que foram extraidos. Há
também a opção de realizar a conversão desses \textit{scripts} para receitas Chef e módulos
Puppets (funcionam como receitas Chef, mas são específicos para a ferramenta Puppet).
Por esse motivo, o Blueprint é considerado como uma ferramenta concorrente ao Cupper,
pois a sua funcionalidade é semelhante.

A tablela~\ref{tab:cupper-blueprint} mostra uma comparação entre o Cupper e Blueprint.

\begin{table}[H]
  \centering
  \begin{tabular}{|l|l|l|}
    \hline
                                                                       & Cupper                                                                                                   & Blueprint                                                                                     \\ \hline
  \begin{tabular}[c]{@{}l@{}}Informações\\ Extraídas\end{tabular}    & \begin{tabular}[c]{@{}l@{}}Pacotes, serviços,\\ configurações, rede,\\ plataforma, hardware\end{tabular} & \begin{tabular}[c]{@{}l@{}}Pacotes, serviços,\\ configurações\end{tabular}                    \\ \hline
  Saídas                                                             & \textit{Cookbook Chef}                                                                                   & \textit{\begin{tabular}[c]{@{}l@{}}Script Shell, Cookbook\\ Chef, Module Puppet\end{tabular}} \\ \hline
  \begin{tabular}[c]{@{}l@{}}Linguagem de\\ Programação\end{tabular} & \textit{Ruby}                                                                                            & \textit{Python}                                                                               \\ \hline
  \begin{tabular}[c]{@{}l@{}}Gerenciador de\\ Pacotes\end{tabular}   & \begin{tabular}[c]{@{}l@{}}APT/dpkg, Pacman,\\ RubyGem e PIP\end{tabular}                                & \begin{tabular}[c]{@{}l@{}}APT, Yum, RubyGems, easy\_install,\\ PIP, PECL e NPM.\end{tabular} \\ \hline
  \end{tabular}
  \caption{Comparativo entre as ferramentas Cupper e Blueprint}
  \label{tab:cupper-blueprint}
\end{table}

É considerado como vantagem do Cupper em relação ao seu concorrente:

\begin{itemize}
  \item O Cupper é focado na ferramenta Chef;
  \item O Cupper segue os padrões adotados pelas outras ferramentas do
    Chef;
\end{itemize}
