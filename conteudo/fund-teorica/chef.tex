\section{Chef}
\label{sec:chef}

Chef é uma ferramenta de gerenciamento e configuração de infraestrutura criada
pela comunidade Opscode em 2008 oficialmente lançada em 2009. Seu propósito é
tranformar uma infraestrutura complexa em código~\cite{chef:2016}. Sendo assim, a gerência de
configuração gira em torno da codificação simplificada e amigável
ao invés de comandos manuais de instalação e configuração de aplicações ~\cite{sharma:2015}.

A estrutura completa da ferramenta Chef contém vários componentes interagindo entre si para
prover ao cliente (chef-client) as informações e instruções necessárias para que ele
possa executar sua função. Os principais componentes são~\cite{chefdoc:2016}:

\begin{itemize}
  \item \textit{Workstation}: qualquer máquina que possa servir como estação
    responsável por permitir usuários criar, testar e manter \textit{cookbooks};
  \item \textit{Cookbook}: contém as configurações desejadas para o ambiente.
    Pode ser customizados para um ambiente específico da empresa ou utilizar
    \textit{cookbooks} disponíveis pela comunidade Chef;
  \item \textit{Ruby}: linguagem oficial utilizada para a escrita dos \textit{scripts}
    é o Ruby;
  \item \textit{Node}: qualquer máquina física, virtual, em núvem, dispositivo
    de rede, etc, que deva ser gerenciada pelo Chef;
  \item \textit{Chef Client}: ferramenta instalada em todos os \textit{nodes}. O
    \textit{Chef Client} é responsável por executar todas as tarefas especificadas
    em uma \textit{run-list} (conjunto de \textit{cookbooks});
  \item \textit{Chef Server}: funciona como um \textit{hub} de informações. Todos os
    \textit{cookbooks} e as políticas são atualizadas no Chef Server pelos usuários
    dos \textit{workstations}. O Chef Client acessa o Chef Server para verificar
    as informações necessárias para sua tarefas e retorna dados para o servidor
    que serão usados para gerar relatórios;
  \item \textit{Chef Analytics}: visibilidade em tempo real de dados informativos
    sobre o servidor como mudanças realizadas, os autores das mudanças e quando
    ocorreram. Além de detalhes das tarefas executadas nas máquinas \textit{nodes};
  \item \textit{Chef Supermarket}: local central onde a comunidade Chef cria e
    mantém os \textit{cookbooks}. Podem ser customizados de acordo com as necessidades
    da organização.
\end{itemize}

