\section{Automação}
\label{sec:auto}

Automação consiste na execução automatica de um processo com o mínimo de
intervensão humana. Para a área de TI, a automação ocorre nos processos de
controle e administração de sistemas ou softwares~\cite{sharma:2015}.

Pode-se citar algumas das vantagens da automação~\cite{sharma:2015}:
\begin{itemize}
  \item Ajuda a reduzir a complexidade de um processo;
  \item Ajuda a reduzir possibilidade de erros humanos em tarefas
    repetitivas;
\end{itemize}

\citeonline{sharma:2015} aborda as necessidade de se adotar automação na área
de TI (Tecnologia da Informação) e relaciona com os conceitos como métodos
ágeis, entrega contínua, computação em nuvem, etc. Além disso, são citados
os benefícios da automação mapeados com as principais preocupações da industria
de TI. Algumas delas:
\begin{itemize}
  \item \textbf{Agilidade}: promove pontualizada de agilidade para a TI. Em conjunto
    com os métodos ágeis resulta em múltiplas implantações em um curto intervalo
    de tempo;
  \item \textbf{Escalabilidade}: a automação ajuda a transformar a infra-estrutura
    em códigos simples, ou seja, a construção, reconstrução e configuração é possível
    ser feita em poucos minutos;
  \item \textbf{Precisão de Implantação}: com a utilização de \textit{scritps}
    é possível realizar rápidas mudanças nas configurações de um ambiente
    obtendo os resultados esperados.
\end{itemize}

A automação, em conjunto com a cultura DevOps, consegue suportar rápidas mudanças,
entrega de contínua, correção de \textit{bugs} com a utilização de código que inclui
vantagens como testes, versionamento de código e integração de aplicações.

