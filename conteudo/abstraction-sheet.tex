\chapter{Abstraction Sheet}

Essa seção de Abstraction Sheet é analoga à etapa de Conduzir entrevistas da fase de Definição do GQM.

\section{Tema 1 - Satisfação do Cliente}

	\subsection{Foco na qualidade}
		\begin{itemize}  
		\item Q1.1
		\item Q1.2
		\item Q1.3
		\end{itemize}
	\subsection{Fatores de variação}
		\begin{itemize}  
		\item Motivação do time de requisitos
		\item Expectativas do Cliente
		\end{itemize}
	\subsection{Hipoteses de baseline}
		\begin{itemize}  
		\item É esperado que a equipe de requisitos comunique-se em linguagem natural com o cliente.
		\item É esperado que o cliente avalie o time com nota 4 em uma escala de 1 a 10.
		\item É esperado que 50\% dos requisito elicitados satisfaçam as necessidades do cliente.
		\end{itemize}
	\subsection{Impactos dos Fatores na BaseLine}
	\begin{itemize}  
		\item O cliente pode ter expectativas muito altas em relação ao produto final, tais que ele não se sinta satisfeito com o entregavel prometido pela disciplina de Engenharia de Requisitos.
		\end{itemize}




\section{Tema 2 - Aderência aos modelos de maturidade}

	\subsection{Foco na qualidade}
		\begin{itemize}  
		\item Q.2.1
		\item Q.2.2
		\end{itemize}
	\subsection{Fatores de variação}
		\begin{itemize}  
		\item Modelo de maturidade usado
		\item Maturidade da equipe
		\end{itemize}
	\subsection{Hipoteses de baseline}
		\begin{itemize}  
		\item É esperado que pelo menos uma atividade de um modelo de maturidade seja agregado
		\end{itemize}
	\subsection{Impactos dos Fatores na BaseLine}
		\begin{itemize}  
		\item A equipe não executar corretamente as atividades aderidas do modelo de maturidade e assim acabar por elicitar incorretamente os requisitos.Aumentando o custo do projeto.
		\end{itemize}


\section{Tema 3 - O tempo gasto para completar as atividades do processo}

	\subsection{Foco na qualidade}
		\begin{itemize}  
		\item Q3.1
		\item Q3.2
		\item Q3.3
		\item Q3.4
		\end{itemize}
	\subsection{Fatores de variação}
		\begin{itemize}  
		\item Motivação do time de requisitos
		\item Preparo do time de requisitos
		\item Desfalque da equipe
		\end{itemize}
	\subsection{Hipoteses de baseline}
		\begin{itemize}  
		\item As atividades do processo serão completas em umtotal de 24 horas quando todo o time estiver trabalhando.
		\end{itemize}
	\subsection{Impactos dos Fatores na BaseLine}
		\begin{itemize}  
		\item O time pode não estar preparado para realizar as atividades propostas e então atrasar a realização das atividades.
		\item Caso algum fator, interno ou externo ao time de requisitos como notas baixas ou motivos pessoais, afete-os a produtividade pode sofrer redução e assim atrasar a realização das atividades
		\end{itemize}


\section{Tema 4 - O custo das atividades}

	\subsection{Foco na qualidade}
		\begin{itemize}  
		\item Q4.1
		\item Q4.2
		\item Q4.3
		\end{itemize}
	\subsection{Fatores de variação}
		\begin{itemize}  
		\item Tempo para completude das atividades
		\item Dedicação do time de requisitos
		\end{itemize}
	\subsection{Hipoteses de baseline}
		\begin{itemize}  
		\item É esperado que o custo de cada atividade seja de CustoHoraAluno*HorasTrabalhadas*NumeroDeAlunos
		\end{itemize}
	\subsection{Impactos dos Fatores na BaseLine}
		\begin{itemize}  
		\item Por fatores diversos, a equipe pode atrasar-se para realizar as atividades, aumentando o custo total no momento da conclusão.
		\end{itemize}