\subsection{Serviço Portal}

A máquina que contem o Portal FGA foi montada com a distro Debian 8 Jessie
e está em operação desde Novembro de 2016. O principal serviço, o site Portal FGA,
é provido pela plataforma Noosfero. A extração com o Cupper foi completa,
sem erros de sintaxe.

O novo ambiente foi montado com a mesma imagem que foi utilizada na máquina
Portal em homologação. Com isso, a distribuição, versão de Kernel e repositórios
de pacotes são as mesmas do ambiente de homologação. Esse novo ambiente, não
continha nenhum serviço ou pacotes instalados que não fossem os padrões
disponível na distribuição.

A Tabela~\ref{tab:result_portal} mostra os principais serviços e pacotes instalados
na máquina em homologação do Portal.

O Cupper foi executado com a opção de versionamento de pacotes desativada. O motivo para
a abordagem é a indisponibilidade dos pacotes nas versões instaladas no ambiente
Portal em homologação. Entretanto o problema persistiu durante a configuração
do novo ambiente, por conta de alguns pacotes não seguirem o padrão de versionamento.
Em geral, os pacotes são instalados especificando a versão: pacote A, versão 1, mas os
pacotes instalados seguia outro padrão: pacote A1, versão 1. Neste caso,
para fazer o \textit{upgrade} do pacote A, é necessário instalar outro pacote: pacote A2,
versão 2. Sendo assim, a receita tentava instalar alguns pacotes em duas versões
diferentes, uma disponível no repositório (atualizada), e outra que não estava mais
no repositório.

Ainda com relação aos pacotes, outro problema foi encontrado. Para instalar pacotes
de um repositório, é necessário uma chave de autenticação para garantir que o repositório
é seguro. O novo ambiente não tem tais chaves, o que causa erro na instalação desses
pacotes.

Tendo em vista esses dois problemas, duas alterações na receita foram necessárias:
remoção dos pacotes sem repositório associado e adição da flag \texttt{allow-unauthenticated}
na instalação dos pacotes não autenticados.


