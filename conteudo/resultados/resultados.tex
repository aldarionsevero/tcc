\newpage\null\thispagestyle{empty}\newpage
\chapter{Resultados}
\label{chap:result}

Neste capítulo será abordado os resultados dos testes realizados
para validar a capacidade do Cupper de reproduzir um ambiente.

Os testes foram realizados em dois tipos de ambientes ao longo do
desenvolvimento do projeto. Os ambientes controlados são máquinas
virtuais locais. Foi utilizado a ferramenta Vagrant e VirtualBox
como virtualizadores e as distros disponíveis no repositório Hashicorp
\footnote{Site oficial que contem todos os tipos de distros em formato \textit{box} para o Vagrant. Disponível em https://atlas.hashicorp.com/boxes/search.}.

Os ambientes reais são máquinas virtuais disponíveis na infraestrutura
do LAPPIS (Laboratório Avançado de Produção Pesquisa e Inovação em Software).
Tais máquinas contem aplicações, serviços e configurações em modo produção.

As proximas sessões irão mostrar o resultado de cada ambiente testado,
a quantidade de pacotes, configurações e serviços que foram replicados
e relatos de problemas ou abordagens.

\section{Premissas e Pré-requisitos}

Os testes podem ser divididos em dois momentos: execução do Cupper no
ambiente alvo e execução da receita gerada pelo Cupper. Portanto,
para os testes, e utilização correta do Cupper, algumas premissas
e pré-requisitos.

No primeiro momento, para a execução do Cupper, o ambiente alvo deve
ser uma distribuição Debian \textit{based}, deve conter Ruby na versão
2.1.0 ou superior, deve ter as depencências para instalação
de Ruby \textit{gems} e o usuário deve ter permissão de execução do Cupper, assim como
permissão de criar pastas e arquivos.

No segundo momento, para execução da receita gerada pelo Cupper, o novo
ambiente deve ser uma distribuição Debian \textit{based} na mesma versão do ambiente
alvo, deve ter acesso aos mesmos repositórios de pacotes que
o ambiente alvo, deve ter acesso a Internet, deve estar configurado com o Chef Client e
o usuário deve ter permissão de instalação de pacotes, criação e alteração de arquivos no
diretório \texttt{/etc}.

Utilizou-se a ferramenta Chake para os testes no segundo momento. O Chake
permite a execução do Chef em um ambiente remoto sem a necessidade de um
Chef Server. O \textit{cookbook} gerado pelo Cupper pode ser utilizado em qualquer
ambiente que utilize o Chef, seja Chef Solo, Chef Zero ou Chef Server.

\section{Ambientes Controlados}

Ambiente Cups
Checklist serviços => importantes
Checklist nova maquina


\section{Ambientes Reais}

1 Checklist serviços => importantes
Checklist nova maquina


1 Checklist serviços => importantes
Checklist nova maquina


1 Checklist serviços => importantes
Checklist nova maquina


