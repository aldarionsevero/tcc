\newpage\null\thispagestyle{empty}\newpage
\chapter{Considerações Finais}
\label{chap:result-parc}

Dividem-se as considerações em duas partes: conclusões
sobre a primeira parte do trabalho e expectativas para
a segunda parte do trabalho.

\textbf{Primeira parte}: nesta primeira parte do trabalho, que
incluem as pesquisa e revisões bibliográficas com relação
as ferramentas de extração e construção de \textit{coookbooks},
conclue-se que é possível realizar a engenharia reversa do
ambiente para a construção de uma receita Chef, visto a
ferramenta Blueprint.

A extração de atributos do ambiente podem ser realizados
com auxílio das ferramentas de sistema e o Ohai que permite
estender com a adição de plugins customizados. Com isto,
aumentar a flexibilidade de extração e filtragem dos dados
de um ambiente.

\textbf{Segunda parte}: espera-se que com a utilização do Cupper,
seja possível replicar um ambiente apenas com a utilização
do Cupper e Chef. O foco principal para a implementação
está na extração dos pacotes, configurações e serviços
de um ambiente.

Não se espera chegar no caso ideal de
recuperação de configurações customizadas, ou seja, fora
dos padrões de disposições dos arquivos de uma aplicação,
mas possivelmente amenizar alguns pontos com o auxilio do
usuário indicando possíveis alterações manuais do ambiente.

\section{Cronograma}
\label{sec:cron}

Esta seção contém o cronograma previsto para a implementação da segunda
parte do trabalho (TCC 2). A visão das atividades são referentes ao
desenvolvimento do objeto proposto pela primeira parte do trabalho (TCC 1)
e pode haver mudanças no decorrer do trabalho.

\begin{table}[H]
  \centering
  \caption{Cronograma TCC 1}
  \label{tab:cron1}
\begin{tabular}{|c|c|c|c|c|c|}
\hline
\rowcolor[HTML]{C0C0C0} 
Atividade                                                              & Março                                           & Abril                    & Maio                                            & Junho                    & Julho                    \\ \hline
Escolha do Tema                                                        & \cellcolor[HTML]{000000}                        &                          &                                                 &                          &                          \\ \hline
\begin{tabular}[c]{@{}c@{}}Levantamento\\ Bibliográfico\end{tabular}   & \cellcolor[HTML]{000000}{\color[HTML]{000000} } & \cellcolor[HTML]{000000} &                                                 &                          &                          \\ \hline
Estudo Inicial                                                         & \cellcolor[HTML]{000000}                        & \cellcolor[HTML]{000000} &                                                 &                          &                          \\ \hline
\begin{tabular}[c]{@{}c@{}}Definição \\ de Objetivos\end{tabular}      &                                                 & \cellcolor[HTML]{000000} & \cellcolor[HTML]{000000}                        &                          &                          \\ \hline
\begin{tabular}[c]{@{}c@{}}Definições\\ Teóricas\end{tabular}          &                                                 & \cellcolor[HTML]{000000} & \cellcolor[HTML]{000000}                        &                          &                          \\ \hline
\begin{tabular}[c]{@{}c@{}}Definições\\ de Metodologia\end{tabular}    &                                                 & \cellcolor[HTML]{000000} & \cellcolor[HTML]{000000}{\color[HTML]{000000} } & \cellcolor[HTML]{000000} &                          \\ \hline
\begin{tabular}[c]{@{}c@{}}Descrição\\ do Desenvolvimento\end{tabular} &                                                 & \cellcolor[HTML]{000000} & \cellcolor[HTML]{000000}                        & \cellcolor[HTML]{000000} &                          \\ \hline
\begin{tabular}[c]{@{}c@{}}Desenvolvimento\\ Inicial\end{tabular}      &                                                 &                          &                                                 & \cellcolor[HTML]{000000} &                          \\ \hline
\begin{tabular}[c]{@{}c@{}}Coleta de Dados\\ Parciais\end{tabular}     &                                                 &                          &                                                 & \cellcolor[HTML]{000000} & \cellcolor[HTML]{000000} \\ \hline
\end{tabular}
\end{table}


\begin{table}[]
\centering
\caption{Cronograma TCC 2}
\label{tab:cron2}
\begin{tabular}{|c|c|c|c|c|c|}
\hline
\rowcolor[HTML]{C0C0C0} 
Atividade                                                                                              & Agosto                   & Setembro                 & Outubro                  & Novembro                 & Dezembro                 \\ \hline
\begin{tabular}[c]{@{}c@{}}Análise\\ Viabilidade\\ Coleta Inicial\end{tabular}                         & \cellcolor[HTML]{000000} &                          &                          &                          &                          \\ \hline
\begin{tabular}[c]{@{}c@{}}Desenvolvimento\\ Funcionalidades\\ Auxiliares\end{tabular}                 & \cellcolor[HTML]{000000} &                          &                          &                          &                          \\ \hline
\begin{tabular}[c]{@{}c@{}}Desenvolvimento\\ da Extração\\ de Atributos\\ Nativos do Ohai\end{tabular} &                          & \cellcolor[HTML]{000000} &                          &                          &                          \\ \hline
\begin{tabular}[c]{@{}c@{}}Desenvolvimento\\ da geração de\\ Relatório\end{tabular}                    &                          & \cellcolor[HTML]{000000} & \cellcolor[HTML]{000000} &                          &                          \\ \hline
\begin{tabular}[c]{@{}c@{}}Desenvolvimento\\ de Plugins\\ Ohai\end{tabular}                            &                          & \cellcolor[HTML]{000000} &                          &                          &                          \\ \hline
\begin{tabular}[c]{@{}c@{}}Desenvolvimento\\ de Geração \\ de Cookbook\end{tabular}                    &                          & \cellcolor[HTML]{000000} & \cellcolor[HTML]{000000} &                          &                          \\ \hline
Coleta de Dados                                                                                        &                          &                          &                          & \cellcolor[HTML]{000000} &                          \\ \hline
Analise de Dados                                                                                       &                          &                          &                          & \cellcolor[HTML]{000000} &                          \\ \hline
\begin{tabular}[c]{@{}c@{}}Aplicação \\ de Melhorias\end{tabular}                                      &                          &                          &                          &                          & \cellcolor[HTML]{000000} \\ \hline
Entrega Final                                                                                          &                          &                          &                          &                          & \cellcolor[HTML]{000000} \\ \hline
\end{tabular}
\end{table}

