\subsection{Serviço Cups}
O ambiente controlado que foi montado para testes iniciais do Cupper foi planejado
para ser simples e conter um serviço com poucas dependências para validar um caso
sem complicações. Diferentemente dos testes tratados a seguir nas próximas Seções,
esse ambiente não se trata de um ambiente real de uso, e foi criado somente para
esse fim.

A máquina virtual utilizada para esse teste foi montada com a distro Debian 8.
O CUPS, que é um serviço de gerenciamento de impressões criado pela Apple foi
instalado nesse ambiente base, juntamente com suas dependências e nginx. A instalação
do CUPS já levanta o seu serviço web que fica ativo no host local na porta 631.
Além de instalar o CUPS, um arquivo de configuração para o CUPS foi criado
para verificar se essa configuração seria passada para o novo ambiente. O
conteúdo desse arquivo pode ser visto no Código~\ref{code:cupsconfig}. Outro 
arquivo de configuração para o Nginx foi definido para efetuar o 
\textit{proxypass} da porta 631 para a porta 80 (Código~\ref{code:nginxconfig}), fazendo assim pelo menos duas
aplicações se relacionarem.

O novo ambiente foi montado com a mesma imagem usada ná máquina virtual já citada.
Os pacotes já instalados e serviços ativos, antes de executar a receita, foram somente 
os que já vem por padrão na distribuição.

O Cupper foi executado com as opções padrão do Cupperfile, e a tabela~\ref{}
mostra os resultados alcançados após gerar a receita no ambiente base e executá-la
no ambiente novo.


\begin{table}[H]
  \centering
  \caption{Resultados da máquina virtual com o serviço Moodle}
  \label{tab:result_moodle}
  \begin{tabular}{c|c|c|c}
    \hline
    \rowcolor[HTML]{EFEFEF} 
    {\color[HTML]{000000} \textbf{Camada}} & {\color[HTML]{000000} \textbf{Ambiente}}                                       & {\color[HTML]{000000} \textbf{Ambiente Replicado}}                             & {\color[HTML]{000000} \textbf{Replicado}} \\ \hline
                                           & \begin{tabular}[c]{@{}c@{}}Dependências do CUPS\\ Instaladas\end{tabular}        & \begin{tabular}[c]{@{}c@{}}Dependências do CUPS\\ Instaladas\end{tabular}        & X                                         \\ \cline{2-4} 
                                           & \begin{tabular}[c]{@{}c@{}}Dependências do Nginx\\ Instaladas\end{tabular}                & \begin{tabular}[c]{@{}c@{}}Dependências do Nginx\\ Instaladas\end{tabular}                & X                                         \\ \cline{2-4} 
                                           & \begin{tabular}[c]{@{}c@{}}Pacote CUPS\\ Instalado\end{tabular} & \begin{tabular}[c]{@{}c@{}}Pacote CUPS\\ Instalado\end{tabular} & X                                         \\ \cline{2-4} 
  \multirow{-6}{*}{Aplicação}            & \begin{tabular}[c]{@{}c@{}}Pacote Nginx\\ Instalado\end{tabular}           & \begin{tabular}[c]{@{}c@{}}Pacote Nginx\\ Instalado\end{tabular}       &                                          X \\ \hline
                                           & \begin{tabular}[c]{@{}c@{}}Serviço Apache\\ Executando\end{tabular}            & \begin{tabular}[c]{@{}c@{}}Serviço Apache\\ Executando\end{tabular}            & X                                         \\ \cline{2-4} 
  \multirow{-2}{*}{Serviços}             & \begin{tabular}[c]{@{}c@{}}Serviço CUPS\\ Executando\end{tabular}             & \begin{tabular}[c]{@{}c@{}}Serviço CUPS\\ Executando\end{tabular}             & X                                         \\ \hline
                                           & \begin{tabular}[c]{@{}c@{}}Configurações CUPS\\ Carregadas\end{tabular}            & \begin{tabular}[c]{@{}c@{}}Configurações CUPS\\ Carregadas\end{tabular}            & X                                         \\ \cline{2-4} 
  \multirow{-2}{*}{Configurações}             & \begin{tabular}[c]{@{}c@{}}Configurações Nginx\\ Carregadas\end{tabular}             & \begin{tabular}[c]{@{}c@{}}Configurações Nginx\\ Carregadas\end{tabular}             & X                                         \\ \hline
  \end{tabular}
\end{table}

No ambiente base em que o CUPS e o Nginx foram instalados manualmente, o CUPS
podia ser acessado na porta 631 do \textit{host} local e o Nginx estava fazendo
com que também fosse possível acessar o CUPS na porta 80. Após a execução da
receita gerada, o novo ambiente passou a ter o mesmo comportamento, e dessa
forma o ambiente foi replicado com sucesso.
