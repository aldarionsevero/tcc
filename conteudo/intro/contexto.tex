\section{Contexto}
\label{sec:contexto}

O mercado de desenvolvimento de software, atualmente, pressiona as organizações
a buscarem por modelos no qual ofereçam uma constante entrega de produto com
um intervalo cada vez menor de tempo. Como parte deste cenário, tem-se o
tradicional problema entre a equipe de desenvolvimento e a equipe de operação
\cite{hummer:2013}.

Enquanto a equipe de desenvolvimento tende a disponibilizar todas novas alterações,
solicitadas pelo cliente ou providas pela empresa, para a validação do cliente o
mais depressa possível, a equipe de operação tende a manter o sistema o estável,
o que significa minimizar a implantação de grandes mudanças. A lacuna desse
processo é agravada pelos diferentes objetivos de cada equipe
\cite{huttermann:2012}.

A adoção de \textit{DevOps} \textit{(Development and Operation)} vem sido feita
para amenizar essa lacuna. \textit{DevOps} consiste em uma série de práticas,
ferramentas ou mesmo cultura organizacional com o objetivo de diminuir o tempo
de entrega e implantação de um software. Juntamente com esse modelo,
tem-se os conceitos de automação de infraestrutura e infraestrutura
como código. Ambos estão fortemente ligados, sendo facilitadores para o
desenvolvedor conhecer as regras de implantação de sua aplicação
e para o operador documentar e configurar um ambiente para um estado específico
definido pelo código de \textit{deployment} \cite{hummer:2013}.

Neste contexto, surgiram ferramentas de suporte, como Chef \cite{chef:2016} e
Puppet \cite{puppet:2016}, que abstraem os passos de execução de configuração
e \textit{deployment} em forma de código. A chave dessas propostas está em torno
da capacidade de \textit{convergence} do ambiente, ou seja, a execução de uma
sequência de passos pré-definidos sempre leva a um estado previsível do objeto
\cite{hummer:2013}. Pode-se citar alguns dos principais benefícios
\cite{vasiliev:2014}:

\begin{itemize}
  \item \textbf{Escalabilidade}: aumentar a quantidade de serviços em uma infraestrutura
    utilizando \textit{environments} (variáveis para ambientes),\textit{roles}
    (conjunto de receitas que representa o papel do sistema) e \textit{nodes}
    (máquina física, virtual, container, etc);
  \item \textbf{Reuso e Replicação}: configurar a mesma aplicação em um novo \textit{node};
  \item \textbf{Documetação}: uma receita Chef contém todas as instruções necessárias
    para configuração do ambiente;
\end{itemize}

