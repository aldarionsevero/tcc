\section{Objetivos}
\label{sec:obj}

\subsection{Objetivo Geral}
\label{sec:obj-grl}

O objetivo deste trabalho é desenvolver uma ferramenta, denominada Cupper,
que extrai, de maneira automatizada, informações de uma máquina para realizar
a construção de \textit{scrips} de automação de infraestrutura. Os \textit{scripts}
serão feitos nos padrões da ferramenta Chef e contém os comandos
necessários para configurar e replicar um ambiente.

\subsection{Objetivos Específicos}
\label{sec:obj-esp}

Os seguintes passos foram levantados para serem seguindos neste trabalho
em conformidade ao proposto no objetivo geral \ref{sec:obj-grl}.

\begin{enumerate}
  \item Identificar as principais e mais relevantes configurações de um sistema
    alvo;
  \item Identificar os componentes estruturais de configuração no padrão da
    ferramenta Chef;
  \item Identificar e definir uma estrutura mínima de configurações Chef;
  \item Mapear quais e onde serão armazenados nos \textit{scripts} os
    diferentes tipos de configurações de um sistema alvo;
  \item Utilizar boas práticas da Engenharia de Software (ex.: testes
    automatizados, modularização e outras) para tornar a ferramenta mais
    manutenível, extensível e reutilizável;
\end{enumerate}
