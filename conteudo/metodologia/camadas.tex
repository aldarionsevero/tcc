%\section{Camadas de Analise do Ambiente}
\section{Análise do Ambiente}
Essa seção irá descrever como será a análise e extração de informações do ambiente. 


%\subsection{Levantamento de camadas a serem analisadas}
\subsection{Definição de Camadas}
Para decidir até que ponto a aplicação irá analisar dentro das configurações
do ambiente é preciso definir os tipos de configuração, as características para
cada um desses tipos, e separar esses tipos em camadas onde a aplicação pode atuar.
Com essas camadas definidas, definem-se quais camadas são relevantes e viáveis
para o escopo do trabalho.

%\subsection{Definição de critérios para selecionar camadas e profundidade de análise}
\subsection{Definição de Critérios para Seleção}
Após levantar as camadas de configuração do ambiente, é necessário definir em
quais o Cupper realmente vai atuar. Os critérios vão estar relacionados a dois
aspectos importantes: relevância para o projeto e quantidade de esforço e tempo
para a implementação (durante o Trabalho de Conclusão de Curso). 

\subsection{Profundidade de Análise}

