%\section{Camadas de Analise do Ambiente}
\section{Análise do Ambiente}
Essa seção irá descrever como será a análise e extração de informações do ambiente. 


%\subsection{Levantamento de camadas a serem analisadas}
\subsection{Definição de Camadas}
Para decidir até que ponto a aplicação irá analisar as configurações
do ambiente é preciso definir os tipos de configuração, as características para
cada um desses tipos, e separar esses tipos em camadas onde a aplicação pode atuar.
Com essas camadas definidas, definem-se quais camadas são relevantes e viáveis
para o escopo do trabalho.

%\subsection{Definição de critérios para selecionar camadas e profundidade de análise}
\subsection{Definição de Critérios para a Seleção}
\label{sec:defcritcamada}
Após levantar as camadas de configuração do ambiente, é necessário definir em
quais camadas e em quais dos seus atributos o Cupper realmente vai atuar. 
Os critérios vão estar relacionados a dois aspectos importantes: a relevância 
para o projeto e a quantidade de esforço e tempo para a implementação 
(durante o Trabalho de Conclusão de Curso).

Para classificar um atributo como relevante para análise, ele deve seguir os
seguintes critérios:

\begin{enumerate}
\item Fazer diferença para o processo de gerar a receita.

Um exemplo básico disso é a arquitetura da CPU, que pode alterar a instalação
e qual a versão de pacotes a serem instalados.
\item Ser informação útil para \textit{logs} e \textit{debugs}, tanto para uso
da ferramenta durante o processo de gerar \textit{cookbooks} quanto para algum
retorno para o usuário, para ajudar a entender possíveis erros ao executar
o \textit{Cupper}.
\item {\color{red} algum mais?}
\end{enumerate}

Com relação a dificuldade de implementação e de integração ao \textit{Cupper} o
atributo deve seguir os seguinte critérios:

\begin{enumerate}
\item É possível implementar a análise do atributo até a segunda parte do
Trabalho de Conclusão de Curso.
\item A implementação da análise do atributo tem complexidade que os
desenvolvedores julgam razoável.
\item O atributo é fornecido como uma das saídas do \textit{Ohai}, facilitando
a sua extração.
\end{enumerate}


{\color{red} se a informação estiver no meio da saída JSON do Ohai, implementação fica mais
simples, então grande influência}

%\subsection{Profundidade de Análise}

