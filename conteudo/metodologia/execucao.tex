\section{Desenvolvimento}

Esta seção descreve o método de desenvolvimento da ferramenta Cupper.
Os itens estão diretamente relacionados com as metodologias, técnicas e
práticas utilizadas na Engenharia de Software, tais como ciclo de
desenvolvimento, testes, integração contínua, qualidade de código, etc.
Todos os itens utilizados fazem parte do conhecimento adquirido durante
o curso e poderão sofrer adaptações de acordo com as necessidades do projeto.

Como será observado nas subseções, as práticas e técnicas são utilizadas
nas Metodologias Ágeis. A motivação para a utilização da abordagem
ágil é a familiaridade e experiência dos desenvolvedores deste trabalho.

\subsection{Métodos Base}

Segundo~\citeonline{gutierrez:2009} o método \textit{Scrum} representa um trabalho
em equipe no qual todos os integrantes e envolvidos trabalham para alcançar
o mesmo objetivo, alinhando as mudanças e compartilhando os problemas para que
todos caminhem para a mesma direção. O mesmo autor descreve o método
\textit{Extreme Programming (XP)} como eficiente, flexível e de baixo risco para equipes
pequenas e médias que convivem com constante mudanças.

Ambos os métodos definem pepeis, práticas e modelos de ciclo de vida para
projetos de desenvolvimentos ágeis.
%TODO: adicionar a descrição de todos os itens e quais a serem utilizados

\subsection{Práticas e Técnicas}

São definidas algumas práticas no método \textit{Scrum}~\cite{gutierrez:2009}. A lista a seguir
mostra as que serão utilizadas, bem como a descrição das adaptações para o trabalho:

\begin{itemize}
  \item \textit{\textbf{Sprint}}: ciclos onde são desenvolvidos os itens propóstos. Geralmente
    são intervalos de 2-4 semanas. Ao final de cada \textit{Sprint} é entregue uma porção 
    executável do \textit{software}. Neste trabalho será utilizado \textit{Sprints} com período
    de 2 semanas;
  \item \textbf{Planejamento da Sprint}: a cada início de \textit{Sprint} é feito uma reunião
    na qual são priorizados os itens a serem desenvolvidos durante a \textit{Sprint}.
    Com o auxílio da ferramenta Git, os itens priorizados para a \textit{Sprint} serão
    postos como \textit{issues} identificadas com a \textit{label} "priority";
  \item \textbf{\textit{Sprint e Product Backlog}}: o \textit{Product Backlog} contém dos os itens a serem
    desenvolvidos no projeto, sendo uma visão macro de tudo a ser feito.
    O \textit{Sprint Backlog} contém os itens a serem desenvolvidos na \textit{Sprint} corrente.
  Com o auxilio da ferramenta Git, ambos os itens dos \textit{backlogs} serão dispostos
  como \textit{issues};
\end{itemize}

O XP traz doze práticas essenciais~\cite{gutierrez:2009}. A lista a seguir
mostra as que serão utilizadas, bem como a descrição das adaptações para o trabalho:

\begin{itemize}
  \item \textbf{Entregas Frequentes}: ao final de cada interação deve haver uma entrega
    dos itens priorizados. Neste trabalho, as integrações serão as (\textit{Sprints})
    e os itens priorizados serão aqueles dispostos no \textit{Sprint Backlog}.
  \item \textbf{Teste}: são divididas em duas partes: teste de aceitação, elaboradas pelo cliente,
    e testes de unidade, elaboradas pelo programador. Será utilizado apenas os testes
    de unidade com a ferramenta RSpec.
  \item \textit{\textbf{Refactoring}}: consiste em simplificar a estrutura, mudar a organização do código,
    sem que altere o comportamento~\cite{beck:2000}. Neste trabalho será utilizado
    conforme a necessidade sendo levado em consideração a importancia, impactos na
    arquitetura da ferramenta e se é prioritário em relação aos outros itens da \textit{Sprint};
  \item \textbf{Integração Contínua}: o código deve ser integrado e testado constantemente
    após o desenvolvimento de novas características. Com o auxílio do serviço Travis CI,
    os \textit{Pull Requests} e \textit{Branchs} serão monitorados quanto a integridade dos testes. Apenas
    será integrado os códigos com teste e que não tenham falhado durante a
    inspeção do serviço de integração contínua;
  \item \textbf{Padrões de Código}: padronização de todo o código para que se apresente
    de forma familiar a toda a equipe, assim facilitando o entendimento. A ferramenta
    Rubocop realiza a análise do código com os padrões da comunidade Ruby, este será
    o padrão adotado neste trabalho.
\end{itemize}


%TODO: adicionar uma exemplificação de todos os itens apresentados como dev
