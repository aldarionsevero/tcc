\section{Coleta e Análise de Resultados}

Nesta seção será apresentado o método de coleta e análise dos resultados deste
trabalho para validação da proposta. Como disposto nos objetivos (\ref{sec:obj}),
o resultado da execução do Cupper é um \textit{script} em formato de receita Chef.
Tal receita poderá ser utilizada pelo Chef para replicar o ambiente. Sendo assim
O foco da coleta e análise de dados está na capacidade de replicar um ambiente
utilizando o Cupper e Chef de maneira automatizada.

Divide-se a coleta e análise de dados em quatro etapas:

\textbf{Etapa 1}: Nesta etapa será coletado os dados sobre quais informações o Cupper consegue
extrair. A coleta será feita por um \textit{checklist} que contenha %TODO: colocar em anexo um checklist?
todos os itens propóstos para implementação da ferramenta. O \textit{checklist} será
utilizado para delimitar os dados base que serão coletados do ambiente para
a validação. Também é um informativo sobre até qual ponto o projeto conseguiu
alcançar.

\textbf{Etapa 2}: Nesta etapa será construído um ambiente que seja possível extraír as configurações
sem a utilização do Cupper. Essas informações estarão alinhadas ao \textit{cehcklist} da etapa
anterior e serão a base para o comparativo com o ambiente replicado.

\textbf{Etapa 3}: Nesta etapa a ferramenta Cupper irá ser executada no ambiente construído na etapa
anterior. As receitas geradas serão usadas pelo Chef em um ambiente limpo que contenha
apenas as configurações mínimas para o seu funcionamento, ou seja, um sistema
operacional (Debian ou Arch), acesso por interface de rede e o Chef (a seção
~\ref{sec:chef} define o ambiente mínimo para o funcionamento do Chef).

\textbf{Etapa 4}: Nesta etapa é feito a extração sem utilização do Cupper, assim como foi realizado
na etapa 2, no ambiente replicado. Então será feito uma comparação das configurações
dos dois ambientes. %TODO: explicar melhor a parte da extração sem o Cupper
