\section{Levantamento dos Recursos Chef}
\label{sec:rec-chef}

A ferramenta Chef provê recursos para total automação da infraestrutura.
Sendo possível preparar, configurar e integrar a infraestrutura com a
flexibilidade de manutenção de scripts \cite{sharma:2015}. Para isso, o
Chef tem um estrutura complexa envolvendo diversos componentes para
o completo funcionamento e utilização de todos os recursos.

De modo análogo ao levantamento de camadas de ambiente,os recursos
de \textit{cookbooks} e do Chef necessários para implementação precisam
ser levantados e então selecionados para o escopo do projeto.

O principal objeto utilizado para a pesquisa do levantamento dos recursos
é a documentação oficial do Chef. A documentação é extensa e completa e
com base nela será feito um levantamento para avaliar quais recursos
serão necessários para a implementação deste projeto. Além disso, existem
referências não oficiais de publicações sobre a ferramenta abordando os
mesmos recursos, entretanto, em sua maioria, demonstram a utilização do
Chef com outras ferramentas de DevOps.

A organização da documentação oficial é dividida nos principais componentes
da arquitetura do Chef além de outros item para conhecimento gerais sobre a
ferramenta \cite{chefdoc:2016}. São eles:

\begin{itemize}
  \item \textit{Getting Started}: visão geral de toda a estrutura do Chef
    como componentes, recursos, ferramentas auxiliares, etc;
  \item \textit{The Workstation}: todos os recursos envolvidos na máquina de \textit{workstation}
    como estrutura, linhas de comando, \textit{kit} de desenvolvimento, etc;
  \item \textit{The Node}: todos os recursos envolvidos nas máquinas \textit{node}
    como atributos do \textit{node}, componentes, etc;
  \item \textit{Cookbook}: explicação de toda a estrutura de um \textit{cookbook}
    como módulos \textit{recipes, templates, files}, etc;
  \item \textit{The Chef Server}: todos os recursos envolvidos na máquina \textit{server}
    como gerenciamento de \textit{nodes}, análise de recursos, \textit{logs}, etc;
  \item \textit{Chef Compliance}: todos os recursos envolvidos na máquina \textit{compliance}
    como informaçõe sobre a infraestrutura, auditoria de configurações, etc.
\end{itemize}

Para o projeto Cupper, os foco principal de estudo dos recursos Chef estão
concentrados em duas seções principais na documentação oficial: \textit{The Node} e
\textit{Cookbook}. As outras seções são complementares para o entendimento de toda
a estrutura e, a princípio, não serão consultadas.

\subsection{Definição de Critérios de Seleção de Recursos}
\label{sec:defcritrecurso}



