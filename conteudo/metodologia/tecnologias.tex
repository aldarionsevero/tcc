\section{Escolha de Tecnologias}
\label{sec:tec}

Já que a aplicação proposta por este trabalho (Cupper) tem a intenção de entrar
na família de ferramentas relacionadas ao Chef, é interessante que ela seja uma
Gem, ou seja, que seja desenvolvida na linguagem Ruby e seja disponibilizada pelo
gerenciador de pacotes RubyGems, estando no padrão das outras ferramentas da
família Chef.

A decisão pelo próprio Chef vem da experiência que o grupo tem com essa ferramenta.
Por já estarem familiarizados com a estrutura, sintaxe e formas de execução do Chef,
a escolha do Chef para a ferramenta de gerência e configuração de ambientes foi a
mais natural.

Por mais que o Chef esteja relacionado ao desenvolvimento do Cupper, ele não é uma
dependência para a execução do mesmo, e sim uma ferramenta que usa a saída do Cupper
para outro fim, como será explicado no capítulo \ref{chap:desenv}, de Desenvolvimento.
Uma dependência que foi levantada para o desenvolvimento do Cupper é o Ohai,
outra ferramenta da família Chef, que será necessária para fazer o perfil e diagnóstico
inicial do sistema.

Para o desenvolvimento, gerência de mudanças, versionamento, testes e integração
contínua serão utilizadas as seguintes ferramentas e \textit{frameworks} respectivamente:
Vim, Git e Github, Rspec e Travis.

{\color{red} Descrição das ferramentas Gitter,Git,Github,Ruby,Rspec
,Gem,Chef,Ohai,Puppet,Travis}

O Git pelo mesmo motivo, e o Github para manter a ferramenta no mesmo serviço de hospedagem
Git de códigos que as outras ferramentas da família Chef. A escolha do Rspec também
tem o objetivo de seguir o mesmo padrão das outras ferramentas da família Chef, que
utilizam Rspec como \textit{framework} de testes. O Travis foi escolhido por ter a
configuração mais simples (em comparação com Jenkins, já que o Travis só requere
um arquivo de configuração e alguns cliques) e também por ser utilizado pela
maioria das ferramentas da família Chef.
