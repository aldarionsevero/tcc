\section{Escolha de Tecnologias}
\label{sec:tec}

%TODO: a seção explica o motívo e critérios utilizados para escolher as tecnologias.

Já que a aplicação proposta por este trabalho (Cupper) tem a intenção de entrar
na família de ferramentas relacionadas ao Chef, é interessante que ela seja uma
Gem, ou seja, que seja desenvolvida na linguagem Ruby e seja disponibilizada pelo %TODO: frase estranha
gerenciador de pacotes RubyGems, estando no padrão das outras ferramentas da
família Chef.

A decisão pelo próprio Chef vem da experiência que o grupo tem com essa ferramenta.
Por já estarem familiarizados com a estrutura, sintaxe e formas de execução do Chef,
a escolha do Chef para a ferramenta de gerência e configuração de ambientes foi a
mais natural. %TODO: essa afirmação é válida?

Por mais que o Chef esteja relacionado ao desenvolvimento do Cupper, ele não é uma
dependência para a execução do mesmo, e sim uma ferramenta que usa a saída do Cupper
para outro fim, como será explicado no capítulo \ref{chap:desenv}, de Desenvolvimento.
Uma dependência que foi levantada para o desenvolvimento do Cupper é o Ohai,
outra ferramenta da família Chef, que será necessária para fazer o perfil e diagnóstico
inicial do sistema. %TODO: melhorar

\section{Ferramentas e Serviços de Suporte ao Desenvolvimento}

Para o desenvolvimento do trabalho, foram escolhidas as ferramentas e serviços de suporte que serão
utilizadas para gerência e controle das atividades do projeto e automação e coleta de dados
de desenvolvimento do Cupper. A escolha foi feita considerando a experiência do
grupo em relação a utilização da ferramenta no decorrer do curso e a identificação dos
padrões utilizados pelas ferramentas providas pelo Chef.

\subsection{Git}

Git é uma ferramenta de controle de versão gratuita e de código aberto criada
em 2005 por Linus Torvalds~\cite{chacon:2014}. O Git foi utilizado para o controle das versões
dos códigos para a produção deste documento e para a produção da ferramenta proposta Cupper.

\subsection{Github}

O Github é um serviço que provê a utilização dos comando Git em um \textit{browser}, além de
disponibilizar um repositório remoto para colaboração de projetos e registro e rastreamento de
\textit{issues}~\cite{github:2016}.  O Github foi utilizado para armazenamento remoto do repositório
deste documento e da ferramenta proposta Cupper, e para o registro e controle de \textit{issues} que são
considerados os itens do \textit{Backlog}, como descrito na seção~\ref{sec:praticas_tecnicas}.

\subsection{Travis}

O Travis é um serviço de integração contínua integrada com o serviço Github que automatiza
a \textit{build} do código e verifica os testes~\cite{travis:2016}. O Travis será utilizado para
realizar a \textit{build} automatica da ferramenta Cupper, determinando se a funcionalidade ou \textit{bugfix}
será integrada ao código.

\subsection{RubyGems}
%TODO: isso se enquadra em ferramentas de suporte?

O RubyGems é um \textit{framework} de empacotamento e instalação de bibliotecas e aplicações
construidas com Ruby. As vantagens de se utilizar RubyGem~\cite{thomas:2001}:

\begin{itemize}
 \item Padronizar formatos de pacotes;
 \item Centralizar o repositório para distribuição dos pacotes Gem;
 \item Facilitar a instalação, gerenciamento e manipulação dos pacotes Gem.
\end{itemize}

A ferramenta Cupper, por ser construída em Ruby, seguirá os padrões adotados
pelo RubyGems.

\subsubsection{RSpec}

O RSpec é um \textit{framework} de \textit{Behaviour-Driven Development} (BDD)
criado por Steven Baker em 2005~\cite{chelimsky:2010}. Com o RSpec
são criados testes que descrevem um comportamento esperado do sistema
em contexto controlados. O RSpec facilita a escrita de testes simplificando
a sintaxe para descrever os cenários e comportamentos.

\subsubsection{Ohai} %TODO: ohai entra como ferramenta de suporte? acredito que seja so a dependencia
\subsubsection{Chef} %TODO: o mesmo vale para o chef?

