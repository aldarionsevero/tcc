\section{Automação}
\label{sec:auto}

Em DevOps, a automação é um elemento essencial para se alcaçar a maturidade
de entregas e \textit{feedback} rápidos. A automação consiste na execução
automatica de um processo com o mínimo de intervensão humana. Para a área
de TI, a automação ocorre nos processos de controle e administração de
sistemas ou softwares~\cite{sharma:2015}.

Pode-se citar algumas das vantagens da automação~\cite{sharma:2015}:
\begin{itemize}
  \item Ajuda a reduzir a complexidade de um processo;
  \item Ajuda a reduzir possibilidade de erros humanos em tarefas
    repetitivas;
\end{itemize}

\citeonline{sharma:2015} aborda as necessidade de se adotar automação na área
de TI (Tecnologia da Informação) e os relaciona com conceitos métodos
ágeis (várias implantações em um curto intervalo de tempo), entrega contínua (várias
\textit{releases} entregues repidamente), computação em nuvem (tendência do mercado a
utilizar infraestrutura em nuvem), etc. Além disso, são citados os benefícios da
automação mapeados com as principais preocupações da industria de TI. Algumas delas:

\begin{itemize}
  \item \textbf{Agilidade}: promove pontualidade e agilidade para a TI. Em conjunto
    com os métodos ágeis resulta em múltiplas implantações em um curto intervalo
    de tempo, além do rápido \textit{feedback};
  \item \textbf{Escalabilidade}: a automação ajuda a transformar a infraestrutura
    em códigos simples, ou seja, a construção, reconstrução e configuração é possível
    ser feita em poucos minutos. Sendo assim é possível manipular grandes quantidades
    de ambientes;
  \item \textbf{Precisão de Implantação}: com a utilização de \textit{scritps}
    é possível realizar rápidas mudanças nas configurações de um ambiente
    obtendo os resultados esperados.
\end{itemize}

A automação, em conjunto com a cultura DevOps, consegue suportar rápidas mudanças,
entrega contínua, correção de \textit{bugs}. Tudo é feito com a utilização de
código que inclui vantagens como testes, versionamento de código e
integração de aplicações~\cite{sharma:2015}.

