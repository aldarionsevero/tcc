\section{Contexto}
\label{sec:contexto}

O mercado de desenvolvimento de software, atualmente, pressiona as organizações
a buscarem por modelos no qual ofereçam uma constante entrega de produto com
um intervalo cada vez menor de tempo. Como parte deste cenário, tem-se o
tradicional problema entre a equipe de desenvolvimento e a equipe de operação
\cite{hummer:2013}.

Enquanto a equipe de desenvolvimento tende a colocar todas novas alterações
para o cliente o mais depressa possível, a equipe de operação tende a manter
o sistema o mais estável possível, o que significa o mínimo de alteração.
A lacuna desse processo é agravada pelos diferentes objetivos de cada equipe
\cite{huttermann:2012}.

A adoção de \textit{DevOps} vem sido feita para amenizar essa lacuna, aplicando
técnicas, ferramentas ou mesmo modificação da cultura. Em conjunto a esse modelo
\textit{DevOps}, tem-se o conceito de automação de infraestrutura e infraestrutura
como código. Ambos estão fortemente ligados, sendo facilitadores para o desenvolvedor
conhecer as regras de \textit{deployment} de sua aplicação e para o operador
documentar e configurar um ambiente para um estado específico definido pelo código
de \textit{deployment} \cite{hummer:2013}.

[Sistema legado]
