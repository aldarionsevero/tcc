\subsection{\textit{Libraries}}

As \textit{libraries} são formas de inserir código Ruby para estender ou definir
uma nova funcionalidade. Semelhante ao que ocorre em \textit{definitions}, entretanto a
capacidade de extensão é maior, sendo possível estender recursos já
existentes~\cite{chefdoc:2016}. Por ser construída em Ruby, todos os
recursos provídos pela linguagem podem ser utilizadas dentro das
\textit{libraries}.

O Código \ref{code:library} mostra a definição para estender um
\textit{recipe} adicionando um método \textit{client} que pode ser utilizado
para facilitar a inserção de dados referente a \textit{ip} e \textit{hostname} de um
\textit{node}

\begin{minipage}{.90\textwidth}
  \lstset{style=shell}
  \lstinputlisting[language=Bash, label=code:library, caption="Exemplo de \textit{libraries}."]{conteudo/code/librarie_example.rb}
\end{minipage}

Em geral, as \textit{libraries} são usadas para definir novas funções que
auxiliam na descrição de rotinas que são mais frequentes durante a
construção de um \textit{recipe} como verificação de arquivos ou serviços.

Esse atributo não será incluido no escopo, pois não é possível definir as \textit{libraries}
necessárias para um contexto desconhecido pelo Cupper.
