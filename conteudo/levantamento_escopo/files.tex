\subsection{\textit{Files}}
\label{sec:cbfiles}

Os \textit{Files} é um recurso referente a manipulação de arquivos no ambiente~\cite{chefdoc:2016}.
Pode-se utilizar os seguintes recursos:

\begin{itemize}
  \item \textit{cookbook\_file}: arquivos que são adicionados ao \textit{node} com base
    nos arquivos presentes no diretório \textit{/files} na raiz do \textit{cookbook};
  \item \textit{file}: manipula arquivos que estão presentes no \textit{node};
  \item \textit{remote\_file}: arquivos são adicionado ao \textit{node} a partir de um
    local remoto;
\end{itemize}

Cada recurso \textit{file} é utilizado para um propósito, mas os seus comportamentos
são similares. O Código \ref{code:file} mostra três maneira diferente de
inserir o arquivo \textit{eth1-conf} no ambiente.

\begin{minipage}{.90\textwidth}
  \lstset{style=shell}
  \lstinputlisting[language=Bash, label=code:file, caption="Exemplo de \textit{file}. Três modos de inserir o arquivo \textit{eth1-conf}."]{conteudo/code/file_example.rb}
\end{minipage}

Esse atributo será incluido no escopo os recursos \textit{file} e \textit{cookbook\_file}.
A leitura de um arquivo de configuração ou de implantação de uma aplicação 
é replicado para a pasta \textit{cookbooks/COOKBOOK\_NAME/files/} de mesmo nome.
O \textit{cookbook\_file} será gerado apartir da coleta de arquivos considerado estáticos,
ou seja, o conteúdo não tem variação com relação a nenhum atributo do ambiente,
por exemplo o atributo IP do ambiente.
