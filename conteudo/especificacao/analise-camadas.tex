%\section{Especificação da Análise em Cada Camada do Ambiente}
\section{Camadas de Ambiente}
\label{sec:cam-amb}

As camadas apresentam um conjunto de aspectos do sistema dos quais contém as
informações necessárias que definem o comportamento específico para o
estado desejado para aquele ambiente. Os atributos de cada camada são variantes
que são consideradas para o \textit{deployment} de uma aplicação. Sendo assim, as
camadas são dependentes para o nível de compatibilidade da configuração.

\subsection{Levantamento de Camadas de Ambiente}
As seguintes camadas foram definidas para representar o estado de configuração do
sistema:
\begin{itemize}
  \item \textit{\textbf{Hardware}}: definições física onde o sistema foi implantado.
    Arquitetura, memória, espaço em disco, etc;
  \item \textit{\textbf{Operation System}}: definições do sistema operacional
    implantado. Distribuição, arquitetura, versão, etc;
  \item \textit{\textbf{Application}}: definições das aplicações instaladas.
    Quais as aplicações instaladas, dependências, etc;
  \item \textit{\textbf{Configuration}}: definições das configurações das
    aplicações. Especificações de \textit{deployment} de aplicação, arquivos
    de configuração;
  \item \textit{\textbf{Service}}: definições dos serviços daemon que estão em
    funcionamento no sistema.
  \item \textit{\textbf{Custom}}: definições criadas específicamente para o
    sistema sem uma forma padrão conhecida.
\end{itemize}

\subsubsection{\textit{Hardware}}
\label{sec:cam-hard}

A camada de \textit{Hardware} contém as definições física do sistema.
As informações são referentes as configurações físicas da máquina, como
cpu, memória, espaço em disco, particionamento, etc. O informe desses atributos
são utilizados para a definição da base do ambiente, ou seja, o sistema
operacional e as aplicações podem ter diferentes desempenhos apartir das
configurações de hardware e/ou apresentar comportamentos inesperados no sistema.
Além das aplicações terem seus requisitos mínimos, algumas são desenhadas
para um tipo específico de arquitetura.

%TODO: adicionar as informações que serão coletadas.

\subsubsection{\textit{Operation System}}
\label{sec:cam-os}

%\subsection{Camadas selecionadas para atuação da aplicação}
\subsection{Seleção de Camadas}
\label{sec:sel-cam}

As camadas foram selecionadas considerando os critérios em~\ref{sec:defcritcamada}
%TODO: completar

\subsection{Profundade de Análise}

